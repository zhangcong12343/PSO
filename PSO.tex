\newcommand{\path}{/Users/chenyu/Documents/Research/PaperEngine}
\documentclass[a4paper,10pt]{article}

\input{\path/paper/mypapersetting}


\newcommand{\FuncOT}{\mathcal{F}_\mathsf{OT}}
\newcommand{\FuncMQRPMT}{\mathcal{F}_\mathsf{mqRPMT}}
\newcommand{\FuncPSU}{\mathcal{F}_\mathsf{PSU}}
\newcommand{\FuncPSI}{\mathcal{F}_\mathsf{PSI}}
\newcommand{\FuncPSO}{\mathcal{F}_\mathsf{PSO}}

\newcommand{\SimR}{\mathsf{Sim}_\mathcal{R}}
\newcommand{\SimS}{\mathsf{Sim}_\mathcal{S}}

\begin{document}

\thispagestyle{empty}

\title{Private Set Operations from Multi-Query Reverse Private Membership Test}

% \author{
%   Min Zhang
%   \thanks{State Key Laboratory of Information Security, Institute of Information Engineering, Chinese Academy of Sciences. School of Cyber Security, University of Chinese Academy of Sciences. Email: \blue{\texttt{zm000min@163.com}}}
%   \and
%   Yu Chen
%   \thanks{School of Cyber Science and Technology, Shandong University, Qingdao 266237, China. Email: \blue{\texttt{yuchen.prc@gmail.com}}}
% }

\date{}

\maketitle

\begin{abstract}
Private set operations allow two parties perform secure computation on two private sets, 
such as intersection or union related functions. 
In this paper, we identify a framework for performing private set operations. 
At the technical core of our framework is multi-query reverse private membership test (mqRPMT), 
which is a natural extension of RPMT recently proposed by Kolesnikov et al.~\cite{KRTW-ASIACRYPT-2019}. 
In mqRPMT, a client with elements vector $(x_1, \dots, x_n)$ interacts with a server holding a set $Y$. 
As a result, the server only learns a bit vector $(e_1, \dots, e_n)$ indicating whether $x_i \in Y$ 
but without knowing the value of $x_i$, while the client learns nothing. 
We first show how to build mqRPMT from a new cryptographic primitive called 
commutative weak  pseudorandom function (cwPRF), 
which in turn can be instantiated from the decisional Diffie-Hellman like assumptions in the random oracle model. 
We then show a relaxed version of mqRPMT can be build from a category of mqPMT 
(which in turn can be based on homomorphic encryption), 
making the first step towards establishing the relation between the two primitives.    

We demonstrate the practicality of our framework with an implementation. 
By plugging our wcPRF-based mqRPMT to the general framework, 
we obtain competitive PSU protocol. 
For input sets of size $2^{20}$, the resulting PSU protocol requires roughly 100 MB bandwidth, 
and 200 seconds using a single thread. 
To the best of our knowledge, it requires the least communication among all the known PSU protocols. 
By plugging our FHE-based mqRPMT$^*$ to the general framework, 
we obtain a PSU$^*$ suitable for unbalanced setting, whose communication complexity is linear in the size of the smaller set, 
and logarithmic in the larger set.    
%Our protocol is amenable to parallelization.  
  
           
\begin{trivlist}
\item \textbf{Keywords:} PSO, PSU, commutative weak PRF, multi-query RPMT
\end{trivlist}
\end{abstract}

\thispagestyle{empty}

\newpage
\setcounter{tocdepth}{2}
\tableofcontents
\newpage

\pagestyle{plain}
\pagenumbering{arabic}
\setcounter{page}{1}
	
	

\section{Introduction}\label{sec:introduction}
Private set operation enables two parties, a sender ($P_1$) and a receiver ($P_2$), with respective input sets $X$ and $Y$ 
to compute the intersection $X \cap Y$ or union $X \cup Y$ without revealing anything else. 
When measuring the efficiency of a PSO protocol, there are two major metrics. 
The first is \emph{computation cost}, which is the amount of computing time necessary to run the protocol. 
Optimizing the computation cost is especially important in practice because of limited computational resources. 
The second is \emph{communication cost},  which is the total amount of communication in the protocol. 
Minimizing the communication cost is also crucial in practice due to limited network bandwidth. 
A hybrid metric is the total running time of the protocol, which includes both the computation time 
and the time to transmit and receive the messages. 
Recent works~\cite{PRTY-CRYPTO-2019, Ion-EUROSP-2020} suggest an alternative efficiency metric -- 
the \emph{monetary cost} to run the protocol 
on a cloud computing service. This new metric takes both computation cost and communication cost into consideration.  

We briefly review PSO protocols in the semi-honest model as below. (add balanced and unbalanced here)

\begin{trivlist}
\item \textbf{Private set intersection.} PSI has found many applications including privacy-preserving sharing, 
    private contact discovery, DNA testing and pattern matching. 
    In the past two decades PSI has been extensively studied and 
    has become truly practical with extremely fast implementation.  
    We refer to~\cite{PSZ-ACM-2018} for a good survey of different PSI paradigms. 
    State-of-the-art two party PSI protocols~\cite{KKRT-CCS-2016, PRTY-CRYPTO-2019, CM-CRYPTO-2020, 
    GPRTY-CRYPTO-2021, RS-EUROCRYPT-2021} mainly rely on symmetric-key operations, 
    except a little public-key operations in base OT used in the OT extension protocol. 

\item \textbf{Private computing on the intersection.} Many real-world applications requires only 
    partial/aggregate information about the intersection to be revealed, 
    such as PSI-card for intersection cardinality~\cite{HFH-EC-1999} 
    and PSI-sum for intersection sum~\cite{Ion-EUROSP-2020, MPRSY-CRYPTO-2020}. 

\item \textbf{Private set union.} Like well-researched PSI, PSU also has numerous applications in practice, 
    such as cyber risk assessment and management via joint IP blacklists and joint vulnerability data. 
    There are two categories of existing PSU protocols. 
    The first is mainly based on public-key techniques~\cite{KS-CRYPTO-2005, Frikken-ACNS-2007, HN-PKC-2010, DC-ACISP-2017}. 
    The second is mainly based on symmetric-key techniques~\cite{KRTW-ASIACRYPT-2019, GMRSS-PKC-2021}.   
    In contrast to the affairs of PSI, the efficiency of the state-of-the-art PSU is less satisfactory. 
    None of existing protocols is optimal in the sense that both communication and computation complexity 
    is linear in the size of sets. 
\end{trivlist} 

PSI, PSI-card/PSI-sum and PSU are closely related functionalities. 
It is somewhat surprising that the state-of-the-art protocols for these functionalities have significantly different efficiency. 
In balanced setting, there is no PSU protocol with linear complexity in the literature.  
\blue{The simplest PSI-card is concretely about $20 \times$ slower and requires over $30 \times$ 
more communication than PSI. 
PSU is concretely about $20 \times$ slower and requires over $30 \times$ more communication than PSI.} 
In unbalanced setting, there is no PSU protocol with sublinear complexity in the larger set in the literature.   
Why is this the case? We observe that the core primitive that underlies almost all PSI protocols 
is a protocol called multi-query private membership test (PMT), which has very efficient realizations. 
However, it seems that mqPMT does not readily implies PSI-card/PSI-sum or PSU. 
The reason is that mqPMT reveals information about intersection, which should be kept privately in PSI-sum/PSI-card and PSU. 

\subsection{Motivation}
The above discussion indicates that the most efficient approach for PSI cannot be used in PSI-card/PSI-sum and PSU. 
Therefore, different approaches are employed in different private set operations, 
creating much more engineering effort. 
We are motivated to seek for the core protocol that enables all private set operations, 
with the hope to deploy PSO in a unified framework.     
Moreover, given the huge efficiency difference between PSI and other closely related protocols, 
we are motivated to give efficient construction of the core protocol to close the gap. 
In summary, we are motivated to answer the following questions:
\begin{center}
\emph{Is there a core protocol that enables a unified framework for all private set operations? 
If so, can we give efficient constructions?} 
\end{center}  

\subsection{Our Contribution}
In this work, we make positive progress on the aforementioned questions. 
We summarize our contribution as below. 

\begin{trivlist}
\item \textbf{A framework of PSO.} We identify that multi-query reverse private membership test (mqRPMT) 
    is a ``Swiss army knife'' for private set operations. 
    More precisely, mqRPMT itself implies PSI-card; by coupling with OTe, mqRPMT implies PSI and PSU; 
    by further coupling with secret sharing or additively homomorphic encryption, mqRPMT implies PSI-sum.  
    Therefore, we can build a PSO framework central around mqRPMT, 
    which can perform operations in a unified and flexible manner.  
    
\item \textbf{Efficient construction of mqRPMT.} We propose a generic construction of mqRPMT with linear complexity
    from a new cryptographic primitive called commutative weak PRF, 
    which in turn can be instantiated from the DDH like assumptions in the random oracle model. 
    The construction is simple and enjoys linear communication and computation complexity. 
    Note that the asymptotic complexity of our PSO framework is dominated by the underlying mqRMPT. 
    Therefore, all protocols derived from our framework achieve linear complexity. 
    \blue{Particularly, to the best of our knowledge, it is the first time to have PSU with linear complexity.}

\item \textbf{Relaxed mqRPMT.} We proprose a relaxed version of mqRPMT (denoted mqRPMT$^*$ hereafter). 
    Compared to the standard mqRPMT, mqRPMT$^*$ allows the sender learn the size of intersection. 
    We show that mqRMPT$^*$ can be build from a special category of mqPMT in a black-box manner 
    via the ``permutate-then-test'' recipe. 
    This makes the initial step towards exploring the connection between mqRPMT and mqPMT. 
    By instantiating the conversion with fully homomorphic encryption (FHE), 
    we obtain an effcient mqRPMT$^*$ in unbalanced setting, 
    which immediately gives rise to a PSU$^*$ protocol in unbalanced setting.      

\item \textbf{Evaluations.} We implement our framework. 
    The experimental results demonstrate that our PSU protocol is superior to all the known PSU protocols 
    in terms of communication cost.   
\end{trivlist}


\subsection{Technical Overview}

\begin{trivlist}
\item \textbf{PSO from mqRPMT.} As discussed above, 
    mqPMT that underlies leading PSI protocols is not applicable for computing PSU as well as other closely related function 
    such as PSI-card and PSI-sum. 
    Consquently, there is no unified framework for PSO.  
    We examine the reverse direction, i.e., whether the core protocol underlying PSU can be used for computing PSI. 
    We identify that the technical core beneath all the existing PSU protocols is mqRPMT, 
    which is a generalization of RPMT proposed in~\cite{KRTW-ASIACRYPT-2019}. 
    Roughly speaking, mqRPMT is a two party protocol between a client with set $X$ and 
    a server $S$ with set $X$. 
    After execution of the protocol, the server learns an indication bit vector $(e_1, \dots, e_n)$ 
    such that $e_i = 1$ if and only if $x_i \in Y$ but without knowning $x_i$, while the client learns nothing. 
    Superficially, mqRPMT is similar to mqPMT, 
    except that the server but not the client learns the test results instead.  
    This tiny difference turns out to be significant. 
    To see this, note that in mqRPMT the information of intersection (except its cardinality) is hidden from both sides, 
    while in mqPMT the intersection is finally known by the client. 
    In light of this difference, mqRPMT is particular suitable for functionalities that have to keep intersection private. 
    A PSU protocol is immediate by having the sender (play the role of client) and the receiver (play the role of server) 
    invoke a mqRPMT protocol on the first place, 
    then carrying out $n$ one-out-two OT with $e_i$ and $(\bot, y_i)$ respectively. 
    PSI and other protocols such as PSI-card and PSI-sum can be constructed similarly 
    by coupling with additively homomorphic encryption or secret sharing. 

\item \textbf{mqRPMT from cwPRFs.} The seminal PSI protocol~\cite{Meadows-SP-1986} 
    (related ideas were appeared in~\cite{Shamir-ICALP-1980, HFH-EC-1999}) 
    is based on the commutative properties of the DH function. 
    After roughly four decades, 
    the DH-based PSI protocol is still the most easily understood and communication efficient one 
    among numerous PSI protocols. It is somewhat surprisingly that no counterpart is known in the PSU setting yet.  
    An intriguing question is: Can DH strike back?

    In this work, we give an affirmative answer. 
    We propose a new cryptographic primitive called commutative weak PRF. 
    Let $F: K \times D \rightarrow R$ be a family of weak PRFs, where $R \subseteq D$. 
    We say $F$ is commutative if for any $k_1, k_2 \in K$ and any $x \in D$, 
    it holds that $F_{k_1}(F_{k_2}(x)) = F_{k_2}(F_{k_1}(x))$. 
    In other words, the two composite functions $F_{k_1} \circ F_{k_2}$ and $F_{k_2} \circ F_{k_1}$ 
    are essentially the same function, say, $\hat{F}$.    

    We then show how to build mqRPMT from cwPRFs.   
    Let server and client generate cwPRF key $k_1$ and $k_2$ respectively, 
    and both of them map their items to elements in the domain $D$ of $F$ 
    via a common cryptographic hash function $\mathsf{H}$, which will be modeled as a random oracle. 
    We begin with the construction of the basic single-query RPMT.
    We first observe that cwPRF immediately gives rise to a private equality test (PEQT) protocol. 
    Suppose server with $y$ and client with $x$, they perform PEQT via the following steps: 
    (1) server computes and sends $F_{k_1}(\mathsf{H}(y))$ to client; 
    (2) client computes and sends $F_{k_2}(\mathsf{H}(x))$ and $F_{k_2}(F_{k_1}(\mathsf{H}(y)))$ to server; 
    (3) $P_1$ then learns the test result by comparing $F_{k_1}(F_{k_2}(\mathsf{H}(x))) =? F_{k_2}(F_{k_1}(\mathsf{H}(y)))$. 
    The commutative property of $F$ guarantee the correctness. 
    The weak pseudorandomness of $F$ guarantee that $P_2$ learns nothing 
    and $P_1$ learns nothing beyond the test result.  
    At a high level, 
    $F_{k_2}(F_{k_1}(\mathsf{H}(\cdot))) = F_{k_1}(F_{k_2}(\mathsf{H}(\cdot))) = \hat{F}(\mathsf{H}(\cdot))$ 
    serves as pseudorandom encoding in the joint view, 
    while $F_{k_1}(\mathsf{H}(\cdot))$ and $F_{k_2}(\mathsf{H}(\cdot))$ 
    serve as partial pseudorandom encoding in the views of server and client respectively.
    
    But, direct extension of the above PEQT protocol by sending back $F_{k_2}(F_{k_1}(\mathsf{H}(y_i)))$ 
    for each $y_i \in Y$ in the same order with server's first move message $F_{k_1}(\mathsf{H}(y_i))$ 
    does not lead to a RPMT protocol. 
    The reason is that $\{\hat{F}(\mathsf{H}(y_i))\}$ serves as an order preserving pseudorandom encoding of set $Y$. 
    As a consequence, the server will learn the exact value of $x$ if $x \in Y$.  
    In order to perform the membership test in an oblivious manner, 
    the idea is to make the pseudorandom encoding of $Y$ independent of the order known by the server. 
    The most straightforward approach is to permutate $\{\hat{F}(\mathsf{H}(y_i))\}$. 
    In this way, we build a single query RMPT protocol from cwPRFs, 
    and the resulting protocol can be easily batched to handle multiple queries 
    by reusing the pseudorandom encoding of $X$. 
    A simple calculation shows that the computation cost is $3n$ times evaluation of $F$ and $n$ times look up, 
    and the communication cost is $3n$ elements in the range of $F$. 
    The mqRPMT protocol is optimal in the sense that both computation and communication complexity 
    is linear to the set size.  
    A more efficient method is to insert $\{\hat{F}(\mathsf{H}(y_i))\}$ into an order hiding data structure 
    such as the Bloom filter, instead of permuting them. 

\item \textbf{mqRPMT$^*$ from mqPMT.} We first abstract a category of PMT protocol called amortized Sigma-PMT 
    with stateless test. Such-PMT naturally give rises to mqPMT, which underlies many PSI protocols with linear complexity. 
    Following the permute-then-recover approach, we can tweak such mqPMT to mqRPMT$^*$ with same asymptotic complexity.  

\item \textbf{Applications of multi-query RPMT.} With multi-query RPMT in hand, we can build a general PSO framework. 
    Multi-query RPMT itself immediately give rise to private set intersection/union cardinality. 
    Coupling with oblivious transfer, we can obtain PSI, private set intersection sum or PSU, 
    depending the messages on OT sender's side. 
\end{trivlist}

\subsection{Related Works}

% In this work, we present a generic construction of PSU protocols from two-message RPMT protocol. 
% We estabilish a connection between PMT and RPMT. 
% More precisely, starting from any reusable two-message PMT protocol, we can convert it to reusable two-message RPMT protocols. 
% By plugging exisiting reusable two-message PMT protocol, we obtain several efficient reusable two-message RPMT protocols, 
% and thus also efficient PSU protocols with desired features. 

% At the technical core of our PSU construction is the reverse private membership test (RPMT). 

% Kolesnikov et al.~\cite{KRTW-ASIACRYPT-2019} identify that existng fast private membership tests, 
% used in leading PSI protocols are not immediately applicable for computing PSU, 
% and a richer PMT of~\cite{CO-SCN-2018} carries $125 \times$ performance penalty. 

% This seemingly simple functionality adjustment (PMT→RPMT) doesn’t seem to be fixable by a small tweak of PMT. 
% This is because the underlying primitive used to implement fast PMT [KKRT16] is a variant of OT extension, 
% and the role of OT receiver naturally belongs to the player with a single-element input y; 
% it is not clear how to amend the protocol to allow (only) the other player to receive the output.
 

% We propose ab efficient construction of RPMT, which serves as the basis of our PSU protocol. 

\section{Preliminaries}\label{sec:preliminaries}

\subsection{MPC in the Semi-honest Model}
We use the standard notion of security in the presence of semi-honest adversaries. 
Let $\Pi$ be a protocol for computing the function $f(x_1, x_2)$, where party $P_i$ has input $x_i$. 
We define security in the following way.
For each party $P$, let $\text{View}_P(x_1, x_2)$ denote the view of party $P$ 
during an honest execution of $\Pi$ on inputs $x_1$ and $x_2$. 
The view consists of $P$'s input, random tape, and all messages exchanged as part of the $\Pi$ protocol. 

\begin{definition}
2-party protocol $\Pi$ securely realizes $f$ in the presence of semi-honest adversaries if 
there exists a simulator $\mathsf{Sim}$ such that for all inputs $x_1, x_2$ and all $i \in \{1,2\}$:
\begin{gather*}
\mathsf{Sim}(i, x_i, f(x_1, x_2)) \approx_c \text{View}_{P_i}(x_1, x_2)
\end{gather*}
\end{definition}
Roughly speaking, a protocol is secure if the party with $x_i$ 
learns no more information other than $f(x_1, x_2)$ and $x_i$.  


\subsection{Private Set Operation}\label{subsec:pso}
PSO is a special case of secure two-party computation. 
% We call the two parties engaging in PSU the \emph{sender} and the \emph{receiver}. 
% The sender holds a set $X$ of size $n_\mathsf{x}$, and the receiver holds a set $Y$ of size $n_\mathsf{y}$. 
% Both sets consist of $\sigma$-bit strings. 
% We always assume the set sizes $n_\mathsf{x}$ and $n_\mathsf{y}$ are public.  
% The ideal PSU functionality (depicted in Figure~\ref{fig:fpso}) computes the union, 
% outputs nothing to the sender, and $X \cup Y$ to the receiver.

\begin{figure}[!hbtp]
\begin{framed}
\begin{minipage}[center]{\textwidth}
\begin{trivlist}
\item \textbf{Parameters:} size of sets $n$.

\item \textbf{Functionality:}
On input $X = \{x_1, \dots, x_n\} \subseteq \{0,1\}^\ell$ (and possibly $V = \{v_1, \dots, v_n\}$) 
from the sender $P_1$ and $Y = \{y_1, \dots, y_n\} \subseteq \{0,1\}^\ell$ from the receiver $P_2$: 
\begin{itemize}
\item \textbf{intersection:} give $X \cap Y$ to the receiver $P_2$.
\item \textbf{union:} give $X \cup Y$ to the receiver $P_2$.
\item \textbf{union$^*$:} give $|X \cap Y|$ to the sender $P_1$ and $X \cup Y$ to the receiver $P_2$.
\item \textbf{intersection cardinality:} give $|X \cap Y|$ to the receiver $P_2$.  
\item \textbf{intersection sum with cardinality:} give $|X \cap Y|$ and $S = \sum_{i: x_i \in Y} v_i$ to the sender. 
\end{itemize}
\end{trivlist}
\end{minipage}
\end{framed}
\caption{Ideal functionality $\FuncPSO$ for PSO}\label{fig:fpso}
\end{figure}

\section{Protocol Building Blocks}

\subsection{Oblivious Transfer}
Oblivious Transfer (OT)~\cite{Rabin-ePrint-2005} is a central cryptographic primitive in the area of secure computation. 
1-out-of-2 OT allows a sender with two input strings $(m_0, m_1)$ and a receiver with an input choice bit $b \in \{0,1\}$. 
As a result of the OT protocol, the receiver learns $m_b$ and neither party learns any additional information. 
Though expensive public-key operations is unavoidable for a single OT,  
a powerful technique called OT extension~\cite{IKNP-CRYPTO-2003, KK-CRYPTO-2013, ALSZ-EUROCRYPT-2015} 
allows one to perform $n$ OTs by only performing $O(\kappa)$ 
public-key operations (where $\kappa$ is the computational security parameter) and $O(n)$ fast symmetric-key operations. 
In Figure~\ref{fig:fot} we formally define the ideal functionality for OT that provides $n$ parallel instances of OT. 

\begin{figure}[!hbtp]
\begin{framed}
\begin{minipage}[center]{\textwidth}
\begin{trivlist}
\item \textbf{Parameters:} number of OT instances $n$; string length $\ell$. 

\item \textbf{Functionality:}
On input $\{(m_{i,0}, m_{i,1})\}_{i \in n}$ from the sender $P_1$ 
where each $m_{i,b} \in \{0,1\}^\ell$, and input $\vec{b} \in \{0,1\}^n$ from the receiver $P_2$: 
\begin{itemize}
\item Give output $(m_{1,b_1}, \dots, m_{n,b_n})$ to the receiver. 
\end{itemize}
\end{trivlist}
\end{minipage}
\end{framed}
\caption{Ideal functionality $\FuncOT$ for OT}\label{fig:fot}
\end{figure}

\subsection{Multi-Query RPMT}
RPMT~\cite{KRTW-ASIACRYPT-2019} refers to a protocol where the client with input $x$ 
interacts with a server holding a set $Y$. 
As a result, the server learns (only) the bit indicating whether $x \in Y$, 
while the client learns nothing about the set $Y$. 
The default notion of RPMT allows the client to query for a single element.
While this procedure can be repeated several times, 
one may seek more efficient solutions allowing the client to make $n$ distinct queries at a reduced cost. 
This generalized notion of $n$-time RPMT is straightforward to define. 
Hereafter, we refer to $n$-time RPMT as multi-query RMPT.  
In Figure~\ref{fig:fmqrpmt} we formally define the ideal functionality for mqRPMT. 
We also define a relaxed version of mqRPMT called mqRPMT$^*$, 
in which the client is given $|X \cap Y|$.  
\begin{figure}[!hbtp]
\begin{framed}
\begin{minipage}[center]{\textwidth}
\begin{trivlist}
\item \textbf{Parameters:} number of RPMT queries $n$. 

\item \textbf{Functionality:}
On input set $Y$ from the server $P_1$ 
and input set $X = (x_1, \dots, x_n) \subseteq \{0,1\}^\ell$ from the client $P_2$: 
\begin{itemize}
\item Give output a vector $\vec{e} = (e_1, \dots, e_n) \in \{0,1\}^n$ to $P_1$, 
    where $e_i = 1$ if $x_i \in Y$ and $e_i = 0$ otherwise. 
    $^*$Also give $|X \cap Y|$ to $P_1$.  
\end{itemize}
\end{trivlist}
\end{minipage}
\end{framed}
\caption{Ideal functionality $\FuncMQRPMT$ for multi-query RPMT}\label{fig:fmqrpmt}
\end{figure}

\section{Commutative Weak Pseudorandom Functions}

\subsection{Definition of Commutative Weak PRF}
We first recall the notion of standard pseudorandom functions (PRFs)~\cite{GGM-JACM-1986}. 
\begin{definition}[PRF]
A family of PRFs consists of three polynomial-time algorithms as follows: 
\begin{itemize}
\item $\mathsf{Setup}(1^\lambda)$: on input a security parameter $\lambda$, 
    outputs public parameters $pp$, which specifies a family of keyed functions $F: K \times D \rightarrow R$.

\item $\mathsf{KeyGen}(pp)$: on input $pp$, outputs a secret key $k \sample K$. 

\item $\mathsf{Eval}(k, x)$: on input $k \in K$ and $x \in D$, outputs $y \leftarrow F(k, x)$. 
    For notation convenience, we will write $F(k, x)$ as $F_k(x)$ interchangeably. 
\end{itemize}

The standard security requirement for PRFs is pseudorandomness. 
\begin{trivlist}
\item \textbf{Pseudorandomness.} Let $\mathcal{A}$ be an adversary against PRFs and define its advantage as:
\begin{displaymath}
    \AdvA =
        \Pr \left[ b = b':~
        \begin{array}{l}
            pp \leftarrow \mathsf{Setup}(1^\lambda);\\
            k \leftarrow \mathsf{KeyGen}(pp);\\
            b \leftarrow \{0,1\}; \\
            b' \leftarrow \mathcal{A}^{\mathcal{O}_\mathsf{ror}(b, \cdot)}(\lambda);
        \end{array} 
        \right] - \frac{1}{2},
\end{displaymath}
where $\mathcal{O}_\mathsf{ror}(0, x) = F_k(x)$, $\mathcal{O}_\mathsf{ror}(1, x) = \mathsf{H}(x)$ 
(here $\mathsf{H}$ is chosen uniformly at random from all the functions 
from $D$ to $R$\footnote{To efficiently simulate access to a uniformly random function $\mathsf{H}$ from $D$ to $R$, 
one may think of a process in which the adversary's queries to $\mathcal{O}_\mathsf{ror}(1, \cdot)$ 
are ``lazily'' answered with independently and randomly chosen elements in $R$, 
while keeping track of the answers so that queries made repeatedly are answered consistently.}). 
Note that $\mathcal{A}$ can adaptively access the oracle $\mathcal{O}_\mathsf{ror}(b, \cdot)$ polynomial many times.
We say that $F$ is pseudorandom if for any PPT adversary its advantage function 
$\AdvA$ is negligible in $\lambda$. We refer to such security as full PRF security. 
\end{trivlist}
\end{definition}

Sometimes the full PRF security is not needed and it is sufficient 
if the function cannot be distinguished from a uniform random one when challenged on random inputs. 
The formalization of such relaxed requirement is \emph{weak pseudorandomness}, 
which is defined the same way as pseudorandomness except that the inputs of 
oracle $\mathcal{O}_\mathsf{ror}(b, \cdot)$ are uniformly chosen from $D$ by the challenger
instead of adversarially chosen by $\mathcal{A}$. 
PRFs that satisfy weak pseudorandomness are referred to as \emph{weak PRFs}.     


\begin{trivlist}
\item \textbf{Composable.} For a family of keyed function $F$, $F$ is 2-composable if $R \subseteq D$, 
    namely, for any $k_1, k_2 \in K$, the function $F_{k_1}(F_{k_2}(\cdot))$ is well-defined. 
    In this work, we are interested in a special case namely $R = D$. 
    %From now on, we will assume $R = D$ for simplicity.   

\item \textbf{Commutative.} For a family of composable keyed function, we say it is commutative if:  
\begin{gather*}
    \forall k_1, k_2 \in K, \forall x \in X: F_{k_1}(F_{k_2}(x)) = F_{k_2}(F_{k_1}(x))
\end{gather*}
\end{trivlist}

It is easy to see that the standard pseudorandomness denies commutative property. 
Consider the following attack against the standard pseudorandomness of $F_k$ as below: 
the adversary $\mathcal{A}$ picks $k' \sample K$, $x \sample D$, 
and then queries the real-or-random oracle at point $F_{k'}(x)$ and point $x$ respectively, 
receiving back responses $y'$ and $y$. $\mathcal{A}$ then outputs `1' iff $F_{k'}(y) = y'$. 
Clearly, $\mathcal{A}$ breaks the pseudorandomness with advantage $1/2 - \mathsf{negl}(\lambda)$.
Provided commutative property exists, the best security we can expect is weak pseudorandomness. 
Looking ahead, weak pseudorandomness and commutative property may co-exist based on some well-studied assumptions. 

\begin{definition}[Commutative Weak PRF]
Let $F$ be a family of keyed functions $K \times D \rightarrow D$.
$F$ is call commutative weak PRF if it satisfies weak pseudorandomness and commutative property simultaneously.   
\end{definition}

\begin{trivlist}
\item \textbf{Further generalization.} Instead of sticking to one family of keyed functions, 
    commutative property can be defined over two families of keyed functions.   
    Let $F$ be a family of weak PRF $K \times D \rightarrow D$, 
    $G$ be a family of weak PRF $S \times D \rightarrow D$. If the following equation holds, 
\begin{gather*}
\forall k \in K, s \in S, \forall x \in X: F_k(G_s(x)) = G_s(F_k(x)) 
\end{gather*}
we say $(F, G)$ is a tuple of commutative weak PRF.  
\end{trivlist}

We note that our notion if commutative weak PRF is similar to but strictly weaker than a previous notion 
called commutative encryption~\cite{AES-SIGMOD-2003}. 
The difference is that cwPRF neither require $F_k$ be a permutation nor $F_k^{-1}$ be efficiently computable. 

\subsection{Instantiations of Commutative Weak PRF}
We present two instantiations of commutative weak PRF. 
The first is built from the well-studied DDH assumption. 
The second is built from weak pseudorandom effective group actions (EGA), 
which in turn can be realized from the CSI-DDH assumption~\cite{KGV-SAC-2020}.  

\begin{trivlist}
\item \textbf{Construction from DDH.} We build a concrete commutative weak PRF from the DDH assumption. 
\begin{itemize}
\item $\mathsf{Setup}(1^\lambda)$: runs $\mathsf{GroupGen}(1^\lambda) \rightarrow (\mathbb{G}, g, p)$, 
    outputs $pp = (\mathbb{G}, g, p)$. 

\item $\mathsf{KeyGen}(pp)$: outputs $k \sample \mathbb{Z}_p$. 
    Each $k$ defines a function from $\mathbb{G}$ to $\mathbb{G}$, 
    which takes $x \in \mathbb{G}$ as input and outputs $x^k$. 
\end{itemize}
It is straightforward to verify that $F$ is commutative. 
The following lemma establishes its pseudorandomness based on the DDH assumption. 

\begin{lemma}
The above construction is weak pseudorandom assuming the hardness of the DDH problem.
\end{lemma} 

\begin{proof}
Towards a tight security reduction, 
we utilize the random self-reducibility of the DDH problem~\cite{NR-FOCS-1995}. 
Let $\mathcal{B}$ be an algorithm against the DDH assumption.  
Given a DDH challenge instance $(g, g^a, g^b, g^c)$ over cyclic group $\mathbb{G} = \langle g \rangle$ of prime order $p$, 
$\mathcal{B}$ interacts with an adversary $\mathcal{A}$
in the weak pseudorandom experiment, with the aim to determine if $c = ab \bmod p$ or not. 

\begin{trivlist}
\item \underline{Setup:} $\mathcal{B}$ sends $pp = (\mathbb{G}, g, p)$ to $\mathcal{A}$. 
    $\mathcal{B}$ implicitly set $a$ as the key of PRF. 

\item \underline{Real-or-random query:} Upon receiving the $i$-th query to oracle $\mathcal{O}_\mathsf{ror}$, 
    $\mathcal{B}$ picks $d_i, e_i \sample \mathbb{Z}_p$, 
    sets the $i$-th random input $x_i := (g^b)^{d_i} \cdot g^{e_i}$, 
    computes $y_i = g^{cd_i+ae_i}$, then sends $(x_i, y_i)$ to $\mathcal{A}$.

\item \underline{Guess:} $\mathcal{A}$ makes a guess $\beta \in \{0,1\}$, 
    where `0' indicates real mode and `1' indicates random mode. 
    $\mathcal{B}$ forwards $\beta$ to its own challenger.    
\end{trivlist}

Clearly, if $c = ab \bmod q$ then $\mathcal{B}$ simulates the real mode perfectly, 
else $\mathcal{B}$ simulates the random mode perfectly. 
Thereby, $\mathcal{B}$ breaks the DDH assumption with the same advantage as 
$\mathcal{A}$ breaks the pseduorandomness of $F$. 
\end{proof}

\item \textbf{Construction from EGA.} We build cwPRF from weak pseudorandom EGA~\cite{AFMP-ASIACRYPT-2020} as below. 
\begin{itemize}
\item $\mathsf{Setup}(1^\lambda)$: generate $pp$ that describes an effective group action $(\mathbb{G}, X, \star)$. 
\item $\mathsf{KeyGen}(pp)$: picks a random group element $g$ from $\mathbb{G}$ as key. 
    Each $g$ defines a function from $X$ to $X$, which takes $x \in X$ as input and outputs $g \star x$.  
\end{itemize}
It is straightforward to verify that when $\mathbb{G}$ is an abelien group, 
the above construction constitutes cwPRF. 
\end{trivlist}

\section{Multi-query RPMT from Commutative Weak PRF}
In Figure~\ref{fig:RPMT-from-cwPRF}, 
we show how to build mqRPMT from cwPRF $F: K \times D \rightarrow D$ 
and cryptographic hash function $\mathsf{H}: \{0,1\}^\ell \rightarrow D$. 

\begin{figure}[!hbtp]
\begin{framed}
\begin{minipage}[center]{\textwidth}
\begin{trivlist}
\item \textbf{Parameters:} 
\begin{itemize}
    \item Common input: $F: K \times D \rightarrow D$, hash function $\mathsf{H}: \{0,1\}^\ell \rightarrow D$. 

    \item Input of server $P_1$: $Y = \{y_1, \dots, y_n\} \subseteq \{0,1\}^\ell$.

    \item Input of client $P_2$: $X = \{x_1, \dots, x_n\} \subseteq \{0,1\}^\ell$.
\end{itemize}

\item \textbf{Protocol:}

\begin{enumerate}
\item $P_1$ picks $k_1 \sample K$, 
    then sends $\{F_{k_1}(\mathsf{H}(y_1)), \dots, F_{k_1}(\mathsf{H}(y_n))\}$ to $P_2$. 

\item $P_2$ picks $k_2 \sample K$, 
    computes and sends $\{F_{k_2}(\mathsf{H}(x_1)), \dots, F_{k_2}(\mathsf{H}(x_n))\}$ to $P_1$; 
    then computes $\{F_{k_2}(F_{k_1}(\mathsf{H}(y_1)), \dots, F_{k_2}(F_{k_1}(\mathsf{H}(y_n))\}$, 
    sends its permutation or Bloom filter $\Gamma$ to $P_1$; 

\item $P_1$ computes $\{F_{k_1}(F_{k_2}(\mathsf{H}(x_1)), \dots, F_{k_2}(F_{k_1}(\mathsf{H}(x_n))\}$, 
    then sets $e_i = 1$ iff $F_{k_1}(F_{k_2}(\mathsf{H}(x_i))) \in \Gamma$.   
\end{enumerate}
\end{trivlist}
\end{minipage}
\end{framed}
\caption{Multi-query RPMT from commutative weak PRF}\label{fig:RPMT-from-cwPRF}
\end{figure} 


\begin{center}
\begin{tikzpicture}
\node[textnode, name=common-input] {$F: K \times D \rightarrow D$, $\mathsf{H}: \{0,1\}^\ell \rightarrow D$};
\node[textnode, name=P1, below of = common-input, xshift=-14em, yshift=-4em] {$P_1$ (server)};
\node[textnode, above of = P1, yshift=2em] {$Y = (y_1, \dots, y_n)$};  
\node[textnode, name=P2, below of = common-input, xshift=14em, yshift=-4em] {$P_2$ (client)}; 
\node[textnode, above of = P2, yshift=2em] {$X = (x_1, \dots, x_n)$};  

\node[textnode, name = k1, below of = P1, yshift=-3em] {$k_1 \sample K$};
\draw ($(P1.east)+(3em, -3em)$) edge[->] node[above] {$\{F_{k_1}(\mathsf{H}(y_i))\}_{i \in [n]}$} 
      ($(P2.west)+(-3em, -3em)$);  

\node[textnode, name = k2, below of = P2, yshift=-6em] {$k_2 \sample K$};
\draw ($(P2.west)+(-3em, -6em)$) edge[->] node[above] {$\{F_{k_2}(\mathsf{H}(x_i))\}_{i \in [n]}$} 
                                          node[below] {$\Gamma \leftarrow \Pi(\{F_{k_2}(F_{k_1}(\mathsf{H}(y_i)\}_{i \in [n]})$
                                          \\\blue{$\Gamma \leftarrow \mathsf{Bloom}(\{F_{k_2}(F_{k_1}(\mathsf{H}(y_i)\}_{i \in [n]})$}} 
                                          ($(P1.east)+(3em, -6em)$);

\node[textnode, name = test, below of = P1, yshift=-6em] {set $e_i = 1$ iff \\ $F_{k_1}(F_{k_2}(\mathsf{H}(x_i))) \in \Gamma$};   
\end{tikzpicture}
\end{center}

\begin{trivlist}
\item \textbf{Correctness.} The above protocol is correct except the case $E$ that 
    $F_{k_1}(F_{k_2}(\mathsf{H}(x))) = F_{k_1}(F_{k_2}(\mathsf{H}(y)))$ for some $x \neq y$ occurs. 
    We further divide $E$ to $E_0$ and $E_1$. 
    $E_0$ denotes the case that $\mathsf{H}(x) = \mathsf{H}(y)$. 
    $E_1$ denotes the case that $\mathsf{H}(x) \neq \mathsf{H}(y)$ 
    but $F_{k_1}(F_{k_2}(\mathsf{H}(x))) = F_{k_1}(F_{k_2}(\mathsf{H}(y)))$, 
    which can further be divided into sub-cases $E_{10}$ — $F_{k_2}(\mathsf{H}(x)) = F_{k_2}(\mathsf{H}(y))$ 
    and $E_{11}$ — $F_{k_2}(\mathsf{H}(x)) \neq F_{k_2}(\mathsf{H}(y))$ but 
    $F_{k_1}(F_{k_2}(\mathsf{H}(x))) = F_{k_1}(F_{k_2}(\mathsf{H}(y)))$. 
    By the collision resistance of $\mathsf{H}$, we have $\Pr[E_0] = 2^{-\sigma}$. 
    By the weak pseudorandomness of $F$, we have $\Pr[E_{10}] = \Pr[E_{11}] = 2^{-\ell}$. 
    Therefore, we have $\Pr[E] \leq \Pr[E_0] + \Pr[E_{10}] + \Pr[E_{11}] = 2^{-\sigma}+2^{-\ell+1}$. 

\end{trivlist} 

\begin{theorem}
The above multi-query RPMT protocol is secure in the semi-honest model assuming $\mathsf{H}$ is a random oracle 
and $F$ is a family of cwPRF.
\end{theorem}

\begin{proof}
We exhibit simulators $\mathsf{Sim}_{P_1}$ and $\mathsf{Sim}_{P_2}$ for simulating corrupt $P_1$ and $P_2$ respectively, 
and argue the indistinguishability of the produced transcript from the real execution. Let $|X \cap Y| = m$. 

\begin{trivlist}
\item \underline{Corrupt client:} $\mathsf{Sim}_{P_2}$ simulates the view of corrupt client $P_2$, 
    which consists of $P_2$'s randomness, input, output and received messages.
 
    We argue the output of $\mathsf{Sim}_{P_2}$ is indistinguishable from the real execution. 
    For this, we formally show the simulation by proceeding the sequence of hybrid transcripts, 
    where $T_0$ is the real view of $P_2$, and $T_1$ is the output of $\mathsf{Sim}_{P_2}$. 

\item $\text{Hybrid}_0$: $\mathsf{Sim}_{P_2}$ chooses the randomness for $P_1$ (i.e., picks $k_1 \sample K$), 
    and simulates with the knowledge of $Y$. 
\begin{itemize}
    \item RO queries: for random oracle query $\langle z_i \rangle$, 
        picks $\alpha_i \sample D$ and set $\mathsf{H}(z_i):=\alpha_i$. 

    \item Let $\beta_i = \mathsf{H}(y_i)$ for each $y_i \in Y$. 
        $\mathsf{Sim}_{P_2}$ outputs $(F_{k_1}(\beta_1), \dots, F_{k_1}(\beta_n))$.     
\end{itemize}
Clearly, $\mathsf{Sim}_{P_2}$'s simulation is identical to the real view of $P_2$. 

\begin{center}
\begin{tikzpicture}
\node[textnode, name = intersection] {$X \cap Y$}; 
\node[ellipsenode, name = Y, left of =intersection, xshift=-1.5em, 
    minimum height = 3em, minimum width=6em, label={[xshift=-1em]$Y$}] {};
\node[ellipsenode, name = X, left of =intersection, xshift=1.5em, 
    minimum height = 3em, minimum width=6em, label={[xshift=1em]$X$}] {}; 

\node[textnode, right of = X, xshift=10em] {$\mathsf{H}(z_i): = \alpha_i \sample D$};
\end{tikzpicture}
\end{center}

\item $\text{Hybrid}_1$: $\mathsf{Sim}_{P_2}$ does not choose the randomness for $P_1$, 
    and simulates without the knowledge of $Y$. 
    It simulates the RO queries the same way as in $\text{Hybrid}_0$, 
    and only changes the simulation of $P_1$'s message.  
\begin{itemize}
    \item $\mathsf{Sim}_{P_2}$ outputs $(s_1, \dots, s_n)$ where $s_i \sample D$.       
\end{itemize}
We argue that the view in $\text{Hybrid}_0$ and $\text{Hybrid}_1$ are computationally indistinguishable. 
More precisely, a PPT adversary $\mathcal{A}$ \red{(with knowledge of $X$ and $Y$)} 
against cwPRF (with secret key $k$) is given $n$ tuples $(\beta_i, s_i)$ where $\beta_i \sample D$, 
and is asked to distinguish if $s_i = F_k(\beta_i)$ or $s_i$ are random values. 
$\mathcal{A}$ implicitly sets $P_1$'s randomness $k_1:=k$. 
\begin{itemize}
    \item RO queries: for random oracle query $\langle z_i \rangle$ where $z_i \notin Y$,  
        picks $\alpha_i \sample D$ and set $\mathsf{H}(z_i):=\alpha_i$; 
        for random oracle query $\langle z_i \rangle$ where $z_i \in Y$, 
        sets $\mathsf{H}(z_i):=\beta_i$.  

    \item $\mathcal{A}$ outputs $(s_1, \dots, s_n)$.      
\end{itemize}

If $s_i = F_k(\beta_i)$ for $i \in [n]$, then $\mathcal{A}$'s simulation is identical to $\text{Hybrid}_0$. 
If $s_i$ are random values, then $\mathcal{A}$'s simulation is identical to $\text{Hybrid}_1$. 
\end{trivlist}

\begin{trivlist}
\item \underline{Corrupt server:} $\mathsf{Sim}_{P_1}$ simulates the view of corrupt server $P_1$, 
    which consists of $P_1$'s randomness, input, output and received messages.
 
    We argue the output of $\mathsf{Sim}_{P_1}$ is indistinguishable from the real execution. 
    For this, we formally show the simulation by proceeding the sequence of hybrid transcripts, 
    where $T_0$ is the real view of $P_1$, and $T_1$ is the output of $\mathsf{Sim}_{P_1}$. 

\item $\text{Hybrid}_0$: $\mathsf{Sim}_{P_1}$ chooses the randomness for $P_2$ (i.e., picks $k_2 \sample K$), 
    and simulates with the knowledge of $X$. 
\begin{itemize}
    \item RO queries: for random oracle query $\langle z_i \rangle$, 
        picks $\alpha_i \sample D$ and sets $\mathsf{H}(z_i):=\alpha_i$. 

    \item $\mathsf{Sim}_{P_1}$ outputs 
        $\{F_{k_2}(\mathsf{H}(x_i))\}_{x_i \in X}$ 
        and $\Gamma \leftarrow \Pi(\{F_{k_2}(F_{k_1}(\mathsf{H}(y_i))\}_{y_i \in Y})$.      
\end{itemize}
Clearly, $\mathsf{Sim}_{P_1}$'s simulation is identical to the real view of $P_1$. 

\begin{center}
\begin{tikzpicture}
\node[textnode, name = intersection] {$X \cap Y$}; 
\node[ellipsenode, name = Y, left of =intersection, xshift=-1.5em, 
    minimum height = 3em, minimum width=6em, label={[xshift=-1em]$Y$}] {};
\node[ellipsenode, name = X, left of =intersection, xshift=1.5em, 
    minimum height = 3em, minimum width=6em, label={[xshift=1em]$X$}] {}; 

\node[textnode, right of = X, xshift=10em] {$\mathsf{H}(z_i): = \alpha_i \sample D$};
\end{tikzpicture}
\end{center}

\item $\text{Hybrid}_1$: $\mathsf{Sim}_{P_1}$ does not choose randomness for $P_1$, 
    and simulates without the knowledge of $X$, 
    but only with the knowledge of $\vec{e} = (e_1, \dots, e_n)$ where $e_i = 1$ iff $x_i \in Y$.  
    Let the Hamming weight of $\vec{e}$ be $m$. It simulates the RO queries the same way as in $\text{Hybrid}_0$, 
    and changes it simulation of $P_2$'s message. 
\begin{itemize}
    \item $\mathsf{Sim}_{P_1}$ picks $v_i \sample D$ for $i \in [n]$ 
        (associated with $F_{k_2}(\mathsf{H}(x_i))$ where $x_i \in X$), 
        outputs $\{v_i\}_{i \in [n]}$; 
        picks $t_j \sample D$ for $j \in [n-m]$ (associated with $F_{k_2}(\mathsf{H}(y_j))$ where $y_j \in Y - X \cap Y$),  
        outputs the permutation of $(\{F_{k_1}(v_i)\}_{e_i = 1}, \{F_{k_1}(t_j)\}_{j \in [n-m]})$.                
\end{itemize}

We argue that the view in $\text{Hybrid}_0$ and $\text{Hybrid}_1$ are computationally indistinguishable. 
More precisely, a PPT adversary $\mathcal{A}$ \red{(with knowledge of $X$ and $Y$)} against cwPRF 
are given $n+m$ tuples $(\alpha_t, s_t)$ where $\alpha_t \sample D$, 
and is asked to determine if $s_t = F_{k}(\alpha_t)$ or random values. 
$\mathcal{A}$ implicitly sets $P_2$'s randomness $k_2:=k$, picks $k_1 \sample K$. 
\begin{itemize}
    \item RO queries: for $z_i \notin X \cup Y$, picks $\beta_i \sample D$ and returns $\mathsf{H}(z_i):=\beta_i$; 
        for $z_i \in X \cup Y$, returns $\mathsf{H}(z_i) := \alpha_i$. 

    \item $\mathcal{A}$ picks out $s_t$ such that $\mathsf{H}(z_t):=\alpha_t$ for $z_t = x_i \in X$ 
        to form $\{v_i\}_{i \in [n]}$, 
        picks out $s_t$ such that $\mathsf{H}(z_t):=\alpha_t$ for $z_t = x_i \in X \backslash Y$ 
        to form $\{w_j\}_{j \in [n-m]}$. 
        Finally, $\mathcal{A}$ outputs $\{v_i\}_{i \in [n]}$ and the permutation or Bloom filter of 
        $(\{F_{k_1}(v_i)\}_{x_i \in X \cap Y}, \{F_{k_1}(w_j)\}_{j \in [n-m]}$.               
\end{itemize}

If $s_i$ are function values, then the simulation is identical to $\text{Hybrid}_0$, 
else it is identical to $\text{Hybrid}_1$. 
\end{trivlist}
This proves the theorem. 
\end{proof}

\begin{remark}
In the above construction of mqRPMT, $P_1$'s message is of the form $F_{k_1}(y_i)$, 
part of $P_2$'s message is of the form $F_{k_2}(x_i)$. 
The common theme is to apply $\mathsf{H}$ and $F_k$ in a cascade way, 
yielding a composite function $F_k \circ \mathsf{H}: \{0,1\}^* \rightarrow D$.   
By leveraging the programmability of $\mathsf{H}$, 
the composite function $F_k \circ \mathsf{H}$ is family of PRF in the random oracle model. 
In essence, random oracle amplifies weak pseudorandomness to standard pseudorandomness.  
\end{remark}


\section{mqRPMT from mqPMT}

\subsection{Private Membership Test}
Private membership test (PMT) protocol~\cite{PSZ-USENIX-2014} is a two-party protocol in which the client with input $x$ 
learns whether or not its item is in the input set $Y$ of the server. 
PMT can be viewed as a special case of private keyword search protocol~\cite{FIPR-TCC-2005} 
by setting the payload as any indication string.
We consider three-message PMT as below, which we refer to Sigma-PMT hereafter. Sigma-PMT proceeds via following pattern.
\begin{enumerate}
\item Server $P_1$ sends the first round message $a$ to sender $P_2$, 
    which is best interpreted as an encoding of $Y$. 
\item Sender $P_2$ sends query $q$ w.r.t. to his item $x$.
\item Server $P_1$ responds with $t$.
\end{enumerate}

After receiving $t$, client $P_2$ can decide if $x \in Y$ by running $\mathsf{Test}(a, x, q, t)$.
The basic notion of Sigma PMT allows the client ($P_2$) to test for a single item. 
While this procedure can be repeated several times, 
one may seed more efficient protocol allowing the client to test $n$ items at a reduced cost. 
We consider such an extension which we refer to as amortized Sigma-PMT, which satisfies the following two properties:

\begin{itemize}
\item \textbf{Reusable:} the first round message is performed by the server ($P_1$) once and for all
\item \textbf{Non-adaptive:} each test query $q_i$ is independent of previous messages
\end{itemize}
Clearly, amortized Sigma-PMT admits parallelization, hence the round complexity is unchanged even when
handling multiple items. Amortized Sigma-PMT may enjoy an additional property:
\begin{itemize}
\item \textbf{Stateless:} the test algorithm can work in a memoryless way, namely, 
    for any xi and associated $(q_i, t_i)$, we have: $\mathsf{Test}(a, x_i, q_i, t_i) = \mathsf{Test}(a, t_i)$.
\end{itemize}
We depict amortized Sigma-PMT that supports stateless testing in Figure~\ref{figure:sigma-PMT}.

\begin{figure}[!hbth]
\begin{framed}
\begin{minipage}[center]{\textwidth}
\begin{center}
\begin{tikzpicture}
    \node [textnode, name=P1] {server\\$P_1(Y)$}; 
    \node [textnode, name=P2, right of =P1, xshift=16em] {client\\$P_2(x)$}; 
  
    \draw ($(P1.east)+(2em, -3em)$) edge[->] node [above]{$a$} ($(P2.west)+(-2em, -3em)$);   

    \draw ($(P2.west)+(-2em, -6em)$) edge[->] node [above] {$q$} ($(P1.east)+(2em, -6em)$); 

    \draw ($(P1.east)+(2em, -9em)$) edge[->] node [above]{$z$} ($(P2.west)+(-2em, -9em)$); 

    \node [textnode, name=test, below of=P2, xshift=2em, yshift=-9em] {$e \leftarrow \mathsf{Test}(a, x, q, z)$}; 
\end{tikzpicture}
\end{center}
\end{minipage}
\end{framed}
\caption{Sigma PMT}\label{figure:sigma-PMT}
\end{figure}


\begin{figure}[!hbth]
\begin{framed}
\begin{minipage}[center]{\textwidth}
\begin{center}
\begin{tikzpicture}
    \node [textnode, name=P1] {server\\$P_1(Y)$}; 
    \node [textnode, name=P2, right of =P1, xshift=16em] {client\\$P_2(X)$}; 
  
    \draw ($(P1.east)+(2em, -3em)$) edge[->] node [above]{$a$} ($(P2.west)+(-2em, -3em)$);   

    \draw ($(P2.west)+(-2em, -6em)$) edge[->] node [above] {$\vec{q}= \{q_1, \dots, q_{n_2}\}$} ($(P1.east)+(2em, -6em)$); 

    \draw ($(P1.east)+(2em, -9em)$) edge[->] node [above]{$\vec{z} = \{z_1, \dots, z_{n_2}\}$} ($(P2.west)+(-2em, -9em)$); 

    \node [textnode, name=test, below of=P2, xshift=2em, yshift=-9em] {$\vec{e} \leftarrow \mathsf{Test}(a, \vec{z})$}; 
\end{tikzpicture}
\end{center}
\end{minipage}
\end{framed}
\caption{mqPMT from Amortized Sigma PMT Supporting Stateless Test}\label{figure:mqPMT-from-sigma-PMT}
\end{figure}

\subsection{Connection to mqRPMT}

\begin{figure}[!hbth]
\begin{framed}
\begin{minipage}[center]{\textwidth}
\begin{center}
\begin{tikzpicture}
    \node [textnode, name=P1] {server\\$P_1(Y)$}; 
    \node [textnode, name=P2, right of =P1, xshift=16em] {client\\$P_2(X)$}; 
  
    \draw ($(P1.east)+(2em, -3em)$) edge[->] node [above]{$a$} ($(P2.west)+(-2em, -3em)$);   

    \draw ($(P2.west)+(-2em, -6em)$) edge[->] node [above] {$\vec{q}= \{q_1, \dots, q_{n_2}\}$} ($(P1.east)+(2em, -6em)$); 

    \node [textnode, name=permutation, below of=P1, xshift=-4em, yshift=-9em] 
        {pick a random permutation \\$\pi: [n_2] \rightarrow [n_2]$}; 

    \draw ($(P1.east)+(2em, -9em)$) edge[->] node [above]{$\vec{z}^* = \{z_{\pi(1)}, \dots, z_{\pi(n_2)}\}$} ($(P2.west)+(-2em, -9em)$); 

    \node [textnode, name=test, below of=P2, xshift=2em, yshift=-9em] {$\vec{e}^* \leftarrow \mathsf{Test}(a, \vec{z}^*)$}; 

    \draw ($(P2.west)+(-2em, -12em)$) edge[->] node [above] {$\vec{e}^*$} ($(P1.east)+(2em, -12em)$); 


    \node [textnode, name=recover, below of=P1, xshift=-4em, yshift=-12em] 
        {$\vec{e} = \{e^*_{\pi^{-1}(i)}\}_{i=1}^{n_2}$}; 
\end{tikzpicture}
\end{center}
\end{minipage}
\end{framed}
\caption{mqRPMT from mqPMT}\label{figure:mqPMT-from-mqPMT}
\end{figure}

\begin{theorem}
The above mqRPMT$^*$ protocol is secure in the semi-honest model assuming the semi-honest security of the starting mqPMT protocol.  
\end{theorem}

\begin{proof}
We exhibit simulators $\mathsf{Sim}_{P_1}$ and $\mathsf{Sim}_{P_2}$ for simulating corrupt server $P_1$ 
and corrupt client $P_2$ respectively. Let $|X \cap Y| = m$. 

\begin{trivlist}
\item \underline{Corrupt client:} $\mathsf{Sim}_{P_2}$ simulates the view of corrupt client $P_2$, 
which consists of $P_2$'s randomness, input, output and received messages.

We argue that the output of $\mathsf{Sim}_{P_2}$ is indistinguishable from the real execution. 
For this, we formally show the simulation by proceeding the sequence of hybrid transcripts, 
where $T_0$ is the real view of $P_2$, and $T_1$ is the output of $\mathsf{Sim}_{P_2}$. 
\item $\mathsf{Hybrid}_0$: $\mathsf{Sim}_{P_2}$ chooses the randomness $r$ for $P_1$, 
    and simulates with the knowledge of $Y$. Clearly, $\mathsf{Sim}_{P_2}$'s simulation is identical to the real view of $P_2$.    

\item $\mathsf{Hybrid}_1$: $\mathsf{Sim}_{P_2}$ does not choose the randomness for $P_1$, 
    and simulates without the knowledge of $Y$. 
    Instead, it invokes the mqPMT's simulator for $P_2$ on his private input $X$ and output $\vec{e}^*$ 
    to generate $(a, \vec{z}^*)$, outputs $(a, \vec{z}^*)$.  

\item \blue{$\mathsf{Hybrid}_2$: $\mathsf{Sim}_{P_2}$ first generate a random indication vector $\vec{t}^*$ 
    with Hamming weight $m = |X \cap Y|$, then invokes the mqPMT's simulator for $P_2$ on his private input $X$ 
    and output $\vec{t}^*$ to generate $(a, \vec{z}^*)$, outputs $(a, \vec{z}^*)$.}  

\item Clearly, the view in $\mathsf{Hybrid}_0$ and $\mathsf{Hybrid}_1$ are computationally indistinguishable 
    based on the semi-honest security of mqPMT on $P_2$'s side. 
    Besides, since both $e^*$ and $t^*$ follow the same distribution, 
    thus the view in $\mathsf{Hybrid}_1$ and $\mathsf{Hybrid}_2$ are identical. 
    In conclusion, the view in $\mathsf{Hybrid}_0$ and $\mathsf{Hybrid}_2$ are computationally indistinguishable. 


\item \underline{Corrupt server:} $\mathsf{Sim}_{P_1}$ simulates the view of corrupt server $P_1$, 
which consists of $P_1$'s randomness, input, output and received messages.

We argue that the output of $\mathsf{Sim}_{P_1}$ is indistinguishable from the real execution. 
For this, we formally show the simulation by proceeding the sequence of hybrid transcripts, 
where $T_0$ is the real view of $P_1$, and $T_1$ is the output of $\mathsf{Sim}_{P_1}$. 

\item $\mathsf{Hybrid}_0$: $\mathsf{Sim}_{P_1}$ chooses the randomness $r$ for $P_2$, 
    and simulates with the knowledge of $X$. Clearly, $\mathsf{Sim}_{P_1}$'s simulation is identical to the real view of $P_1$.    

\item $\mathsf{Hybrid}_1$: $\mathsf{Sim}_{P_1}$ does not choose the randomness for $P_2$, 
    and simulates without the knowledge of $X$. Instead, it first invokes the mqPMT's simulator for $P_2$ 
    on his input $Y$ to generate $\vec{q}$, then picks a random permutation $\pi$, computes $\vec{e}^* = \pi(\vec{e})$,  
    outputs $(\vec{q}, \pi(\vec{z}))$.  

\item Clearly, the view in $\mathsf{Hybrid}_0$ and $\mathsf{Hybrid}_1$ are computationally indistinguishable 
    based on the semi-honest security of mqPMT on $P_1$'s side. 
\end{trivlist} 
This proves the theorem. 
\end{proof}



\section{Applications of Multi-Query RPMT}
We show how to build a PSO framework central around mqRPMT in Figure~\ref{fig:PSO-from-mqRPMT}.
\begin{figure}[!hbtp]
\begin{framed}
\begin{minipage}[center]{\textwidth}
\begin{trivlist}
\item \textbf{Parameters:} 
\begin{itemize}
    \item Input of receiver $P_1$: $Y = \{y_1, \dots, y_{n_1}\} \subseteq \{0,1\}^\ell$.

    \item Input of sender $P_2$: $X = \{x_1, \dots, x_{n_2}\} \subseteq \{0,1\}^\ell$ 
        and $V = \{v_1, \dots, v_{n_2}\}$.
\end{itemize}

\item \textbf{Protocol:}

\begin{enumerate}
\item $P_1$ (playing the role of server) with $Y$ and $P_2$ (playing the role of receiver) with $X$
    invoke $\FuncMQRPMT$. $P_1$ obtains an indication bit vector $\vec{e} = (e_1, \dots, e_{n_2})$. 
    $P_2$ obtains nothing.  

    \begin{itemize}
        \item \textbf{cardinality:} $P_1$ learns the cardinality by calculating the Hamming weight of $\vec{e}$. 
    \end{itemize}

\item $P_1$ and $P_2$ invoke $n_2$ instances of OT via $\FuncOT$. 
    $P_1$ uses $\vec{e}$ as the choice bits. 


\begin{itemize}
    \item \textbf{intersection:} $P_2$ uses $(\bot, x_i)$ as input to the $i$th OT. 
        $P_1$ learns $\{x_i \mid e_i = 1\}_{i \in [n_2]} = X \cap Y$. 

    \item \textbf{union:} $P_2$ uses $(x_i, \bot)$ as input to the $i$th OT. 
        $P_1$ learns $\{x_i \mid e_i = 0\}_{i \in [n_2]} = X \slash Y$, and outputs $\{X \slash Y\} \cup Y = X \cup Y$. 

    \item \textbf{intersection sum with cardinality:} $P_2$ randomly generate $r_i$ 
        subject to the constraint $\sum_{i=1}^{n_2} r_i = 0$, then uses $(r_i, r_i + v_i)$ as input to the $i$th OT. 
        $P_1$ learns $\{\sum_{i=1}^{n_2} v_i \mid e_i = 1\}_{i \in [n_2]} = X \cap Y$. 
\end{itemize} 

\end{enumerate}
\end{trivlist}
\end{minipage}
\end{framed}
\caption{PSO from mqRPMT}\label{fig:PSO-from-mqRPMT}
\end{figure} 

\begin{theorem}
The resulting PSU protocol is semi-honest secure by assuming the semi-honest security of mqRPMT and OT.
\end{theorem}

\blue{How to formally prove this theorem? Shall we interpret client's 
private input in mqRPMT as a set $X$ or as a vector?}

\cyan{The input of client is absolutely a set $X$ instead of a vector, that is, there is no order in elements of $X$. In other word, the same elements with different order form the same set.  However, one may ask, when we actually excute a protocol, we must use elements in $X$ under some order, how to understand this order? My answer is that, this order just reflects the order in which the protocol is executed, and has no relation with the set itself. In other word, if we execute the protocol with the same set $X$, but under different order, there is no influence in protocol's correctness and security. For example, in PSU, the receiver obtain $x$ in $i$th OT, we should not regard this $i$ leak the information about $X$, because $X$ has no order, in the next execution, this $x$ may be appears in $j$th OT. Both execution is a secure execution of PSU. If one still think the original order of $X$ will leak some information (e.g. in real world, the order may has some deviation), we can just assume all parties select a random permutation to shuffle their input set before they execute any PSO protocols.}

\begin{proof}
\begin{trivlist}
\item \underline{Corrupt Sender:} $\mathsf{Sim}_2(X=\{x_1,\dots,x_{n_2}\})$ 
    simulates the view of corrupt semi-honest sender. It executes as follows:
\begin{enumerate}
\item $\mathsf{Sim}_2$ invokes mq-RPMT simulator $\mathsf{Sim}_\mathsf{mq\text{-}rpmt}^S(X)$ 
    and appends the output to the view.

\item For $i\in [n_2]$, $\mathsf{Sim}_{2}$ invokes OT simulator $\mathsf{Sim}_\mathsf{ot}^S(x_i,\perp)$ 
    and appends the output to the view.
\end{enumerate}
Now we argue that the view output by $\mathsf{Sim}_{2}$ is indistinguishable from the real one. 
This is obtained by the underlying simulators' indistinguishability directly.

\item \underline{Corrupt Receiver:} $\mathsf{Sim}_1(Y=\{y_1,\dots,y_{n_{1}}\}, X \cup Y)$ 
    simulates the view of corrupt semi-honest receiver. It executes as follows:
\begin{enumerate}
\item $\mathsf{Sim}_1$ defines the set $Z := Y\cup X\setminus Y$, 
    i.e. the set of elements that $X$ ``brings to the union''. 
    Next, it uses $\perp$ to pads $Z$ to $n_{2}$ elements and permutates these elements randomly. 
    Let $Z=\{z_1,\dots,z_{n_2}\}$. 

\item $\mathsf{Sim}_1$ sets $e_i = 1$ if and only if $z_i\in Y$ for $i\in [n_2]$. 
    Then, it invokes mq-RPMT simulator $\mathsf{Sim}_\mathsf{mq\text{-}rpmt}^R(Y,e)$ and appends the output to the view.

\item For $i\in [n_2]$, $\mathsf{Sim}_1$ invokes OT simulator 
    $\mathsf{Sim}_\mathsf{ot}^R(e_i,z_i)$ and appends the output to the view.
\end{enumerate}
\end{trivlist}
Now we argue that the view output by $\mathsf{Sim}_{1}$ is indistinguishable from the real one. In the simulation, the way $P_1$ obtains the elements in $Z= X\setminus Y$ is identical to the real execution.\footnote{In fact, the order that $P_1$ obtains $x_i$ is different from real execution. However, this makes no influence in the security of protocol since there is no order in set $X$, the index $i$ just reflects the order of this execution.} By the underlying simulators' indistinguishability, the simulated view is computationally indistinguishable from the real one. 

\end{proof}


\section{Implementation}
\begin{table}[H]
\begin{center}
\caption{The computation and communication complexity of PSU}
\begin{tabular}{|c|c|c|c|c|c|}
\hline
         & \multirow{2}*{Protocol} & Setting  & \multicolumn{3}{c|}{Set size $n$} \\
                                              \cline{4-6} 
         &                         & (LAN)   & $2^{12}$ & $2^{16}$ & $2^{20}$   \\
\hline
\multirow{3}*{Time (s)} & \multirow{3}*{Ours} & 1 Mbps     & & & \\
                        \cline{3-6} 
                        &                     & 10 Mbps    & & & \\
                        \cline{3-6} 
                        &                     & 100 Mbps   & & & \\
                        \cline{3-6} 
                        &                     & 68 Gbps    & 0.45 & 5.3 & 86.9 \\
\hline
Comm. (MB)              & Ours                &            & 0.47 & 7.4 & 117.6\\
\hline

    % confidential transaction &~& $(2 \log_2(\ell) + 20)|\mathbb{G}| + 10|\mathbb{Z}_p|$ &~&1310   &\qquad& 40 & ~ & 14\\
    % \hline 
    % \hline
    % \multirow{2}*{auditing policies} &~& \multicolumn{3}{c}{proof size} &\qquad& \multicolumn{3}{c}{auditing cost (ms)}\\
    %     \cline{2-4}  \cline{5-9} 
    %     &~&~big-$\mathcal{O}$~ &~&~bytes~&\qquad& ~generation~ & ~ &~verify\\
    % \hline
    % limit policy &~& $(2 \log_2(\ell) + 4)|\mathbb{G}| + 5|\mathbb{Z}_p|$ &~&622   &\qquad& 21.5 & ~ & 7.5\\
    % \hline
    % rate policy &~& $2 |\mathbb{G}| + 1|\mathbb{Z}_p|$ &~&98   &\qquad& 0.55 & ~ & 0.69\\
    % \hline
    % open policy &~& $2 |\mathbb{G}| + 1|\mathbb{Z}_p|$ &~&98   &\qquad& 0.26 & ~ & 0.42\\
    % \hline
\end{tabular}{}\label{table:efficiency}
\end{center}
\begin{tablenotes}
\item [a] Here
\end{tablenotes}
\end{table}

and collect the benchmarks on a MacBook Pro with an 2.6GHz Intel CPU and 16GB of RAM

          

\bibliographystyle{alpha}
\bibliography{\path/bib/mycrypto}

\appendix

\section{Missing Definitions}
\subsection{Weak Pseudorandom EGA}
We begin by recalling the definition of a group action. 
\begin{definition}[Group Actions]
A group $\mathbb{G}$ is said to \emph{act on} a set $X$ if there is a map $\star: \mathbb{G} \times X \rightarrow X$ 
that satisfies the following two properties: 
\begin{enumerate}
    \item Identity: if $e$ is the identity element of $\mathbb{G}$, then for any $x \in X$, we have $e \star x = x$. 
    \item Compatibility: for any $g, h \in \mathbb{G}$ and any $x \in X$, 
        we have $(gh) \star x = g \star (h \star x)$. 
\end{enumerate}
\end{definition}

From now on, we use the abbreviated notation $(\mathbb{G}, X, \star)$ to denote a group action. 
We then define an effective group action (EGA)~\cite{AFMP-ASIACRYPT-2020} as follows. 
\begin{definition}[Effective Group Actions]
A group action $(\mathbb{G}, X, \star)$ is \emph{effective} a set $X$ if the following properties are satisfied:  
\begin{enumerate}
\item The group $\mathbb{G}$ is finite and there exist PPT algorithms for: 
\begin{enumerate}
    \item Membership testing, i.e., to decide if a given bit string represents a valid group element in $\mathbb{G}$. 
    \item Equality testing, i.e., to decide if two bit strings represents the same group element in $\mathbb{G}$.
    \item Sampling, i.e., to sample an element $g$ from a uniform (or statistically close to) distritution on $\mathbb{G}$.  
    \item Operation, i.e., to compute $gh$ for any $g, h \in \mathbb{G}$. 
    \item Inversion, i.e., to compute $g^{-1}$ for any $g \in \mathbb{G}$.
\end{enumerate}

\item The set $X$ is finite and there exist PPT algorithms for: 
\begin{enumerate}
    \item Membership testing, i.e., to decide if a bit string represents a valid set element. 

    \item Unique representation, i.e., given any arbitrary set element $x \in X$, 
        compute a string $\hat{x}$ that canonically represents $x$.
\end{enumerate}

\item There exists a distinguished element $x_0 \in X$, 
    called the origin, such that its bit-string representation is known.

\item There exists an efficient algorithm that given (some bit-string representations of) any $g \in \mathbb{G}$ 
    and any $x \in X$, outputs $g \star x$.
\end{enumerate} 
\end{definition}

\begin{definition}[Weak Pseudorandom EGA]
There is no PPT adversary that can distinguish tuples of the form $(x_i, g \star x_i)$ 
from $(x_i, u_i)$ where $g \leftarrow \mathbb{G}$ and each $x_i, u_i \leftarrow X$ are sampled uniformly at random.
\end{definition}

	
\end{document}

% \section{PSU from Commutative Weak PRFs}
% \begin{figure}[!hbtp]
% \begin{framed}
% \begin{minipage}[center]{\textwidth}
% \begin{trivlist}
% \item \textbf{Parameters:} 
% \begin{itemize}
%     \item Common input: $F: K \times D \rightarrow D$, hash function $\mathsf{H}: \{0,1\}^* \rightarrow D$. 

%     \item Input of sender $\mathcal{S}$: $X = \{x_1, \dots, x_n\}$.

%     \item Input of receiver $\mathcal{R}$: $Y = \{y_1, \dots, y_n\}$. 
% \end{itemize}

% \item \textbf{Protocol:}

% \begin{enumerate}
% \item $\mathcal{R}$ picks $k_2 \sample K$, 
%     then sends $\{F_{k_2}(\mathsf{H}(y_1)), \dots, F_{k_2}(\mathsf{H}(y_n))\}$ to $\mathcal{S}$. 

% \item $\mathcal{S}$ picks $k_1 \sample K$, 
%     computes and sends $\{F_{k_1}(\mathsf{H}(x_1)), \dots, F_{k_1}(\mathsf{H}(x_n))\}$ to $\mathcal{R}$; 
%     then computes $\{F_{k_1}(F_{k_2}(\mathsf{H}(y_1)), \dots, F_{k_1}(F_{k_2}(\mathsf{H}(y_n))\}$, 
%     sends its permutation $\Gamma$ to $\mathcal{R}$; 

% \item $\mathcal{R}$ computes $\{F_{k_2}(F_{k_1}(\mathsf{H}(x_1)), \dots, F_{k_2}(F_{k_1}(\mathsf{H}(x_n))\}$, 
%     then sets $v_i = 0$ iff the value is not in $\Gamma$. 

% \item $\mathcal{R}$ with select vector $(v_1, \dots, v_n)$ and $\mathcal{S}$ 
%     with input $\{(x_i, \bot)\}_{i \in [n]}$ engage in one-sided OT. 

% \item $\mathcal{R}$ obtains $X - X \cap Y$, and thus obtains the union $X \cup Y$.  
% \end{enumerate}
% \end{trivlist}
% \end{minipage}
% \end{framed}
% \caption{PSU from Commutative Weak PRFs}\label{fig:PSU-from-cwPRF}
% \end{figure} 

% \begin{trivlist}
% \item \textbf{Correctness.} The above protocol is correct except the case $E$ that 
%     $F_{k_1}(F_{k_2}(\mathsf{H}(x))) = F_{k_1}(F_{k_2}(\mathsf{H}(y)))$ for some $x \neq y$ occurs. 
%     We further divide $E$ to $E_0$ and $E_1$. 
%     $E_0$ denotes the case that $\mathsf{H}(x) = \mathsf{H}(y)$. 
%     $E_1$ denotes the case that $\mathsf{H}(x) \neq \mathsf{H}(y)$ 
%     but $F_{k_1}(F_{k_2}(\mathsf{H}(x))) = F_{k_1}(F_{k_2}(\mathsf{H}(y)))$, 
%     which can further be divided into sub-cases $E_{10}$ — $F_{k_2}(\mathsf{H}(x)) = F_{k_2}(\mathsf{H}(y))$ 
%     and $E_{11}$ — $F_{k_2}(\mathsf{H}(x)) \neq F_{k_2}(\mathsf{H}(y))$ but 
%     $F_{k_1}(F_{k_2}(\mathsf{H}(x))) = F_{k_1}(F_{k_2}(\mathsf{H}(y)))$. 
%     By the collision resistance of $\mathsf{H}$, we have $\Pr[E_0] = 2^{-\sigma}$. 
%     By the weak pseudorandomness of $F$, we have $\Pr[E_{10}] = \Pr[E_{11}] = 2^{-\ell}$. 
%     Therefore, we have $\Pr[E] \leq \Pr[E_0] + \Pr[E_{10}] + \Pr[E_{11}] = 2^{-\sigma}+2^{-\ell+1}$. 

% \end{trivlist} 

% \begin{theorem}
% The above PSU protocol is secure in the semi-honest model assuming $\mathsf{H}$ is a random oracle 
% and $F$ is a family of commutative weak PRFs.
% \end{theorem}

% \begin{proof}
% We exhibit simulators $\SimR$ and $\SimS$ for simulating corrupt $\mathcal{R}$ and $\mathcal{S}$ respectively, 
% and argue the indistinguishability of the produced transcript from the real execution. Let $|X \cap Y| = m$. 

% \begin{trivlist}
% \item \underline{Corrupt sender:} $\SimS$ simulates the view of corrupt $\mathcal{S}$, 
%     which consists of $\mathcal{S}$'s randomness, input, output and received messages.
 
%     We argue the output of $\SimS$ is indistinguishable from the real execution. 
%     For this, we formally show the simulation by proceeding the sequence of hybrid transcripts, 
%     where $T_0$ is the real view of $\mathcal{S}$, and $T_2$ is the output of $\SimS$. 

% \item $\text{Hybrid}_0$: $\SimS$ simulates with the knowledge of $Y$. 
% \begin{itemize}
%     \item $\SimS$ chooses the randomness for $\mathcal{R}$, i.e., picks $k_2 \sample K$.
    
%     \item RO queries: for random oracle query $\langle z \rangle$, 
%         picks $h \sample D$ and set $\mathsf{H}(z):=h$. 

%     \item Let $h_i = \mathsf{H}(y_i)$ for each $y_i \in Y$. $\SimS$ outputs $(F_{k_2}(h_1), \dots, F_{k_2}(h_n))$.     
% \end{itemize}
% Clearly, $\SimS$'s simulation is identical to the real view. 

% \begin{center}
% \begin{tikzpicture}
% \node[textnode, name = intersection] {$X \cap Y$}; 
% \node[ellipsenode, name = X, left of =intersection, xshift=-1.5em, 
%     minimum height = 3em, minimum width=6em, label={[xshift=-1em]$X$}] {};
% \node[ellipsenode, name = Y, left of =intersection, xshift=1.5em, 
%     minimum height = 3em, minimum width=6em, label={[xshift=1em]$Y$}] {}; 

% \node[textnode, right of = Y, xshift=10em] {$\mathsf{H}(z_i): = h_i \sample D$};
% \end{tikzpicture}
% \end{center}

% \item $\text{Hybrid}_1$: $\SimS$ simulates without the knowledge of $Y$, 
%     and changes the simulation in the final input.  
% \begin{itemize}
%     \item $\SimS$ outputs $(s_1, \dots, s_n)$ where $s_i \sample D$.       
% \end{itemize}
% We argue that the view in hybrid 0 and hybrid 1 are computationally indistinguishable. 
% More precisely, a PPT adversary $\mathcal{A}$ \red{(with knowledge of $X$ and $Y$)} 
% against commutative weak PRF (with secret key $k$) is given $n$ tuples $(h_i, s_i)$ where $h_i \sample D$, 
% and is asked to distinguish if $s_i = F_k(h_i)$ or $s_i$ are random values. 
% $\mathcal{A}$ implicitly sets $\mathcal{R}$'s randomness $k_2:=k$. 
% \begin{itemize}
%     \item RO queries: for random oracle query $\langle z \rangle$ where $z \notin Y$,  
%         picks $h \sample D$ and set $\mathsf{H}(z):=h$; for random oracle query $\langle y_i \rangle$ where $y_i \in Y$, 
%         sets $\mathsf{H}(y_i):=h_i$.  

%     \item $\mathcal{A}$ outputs $(s_1, \dots, s_n)$.      
% \end{itemize}

% If $s_i = F_k(h_i)$ for $i \in [n]$, then $\mathcal{A}$'s simulation is identical to hybrid 0. 
% If $s_i$ are random values, then $\mathcal{A}$'s simulation is identical to hybrid 1. 
% \end{trivlist}

% \begin{trivlist}
% \item \underline{Corrupt receiver:} $\SimR$ simulates the view of corrupt $\mathcal{R}$, 
%     which consists of $\mathcal{R}$'s randomness, input, output and received messages.
 
%     We argue the output of $\SimR$ is indistinguishable from the real execution. 
%     For this, we formally show the simulation by proceeding the sequence of hybrid transcripts, 
%     where $T_0$ is the real view of $\mathcal{R}$, and $T_2$ is the output of $\SimR$. 

% \item $\text{Hybrid}_0$: $\SimR$ simulates with the knowledge of $X$. 
% \begin{itemize}
%     \item $\SimR$ chooses the randomness for $\mathcal{S}$, i.e., picks $k_1 \sample K$.
    
%     \item RO queries: for random oracle query $\langle z \rangle$, picks $h \sample D$ and sets $\mathsf{H}(z):=h$. 

%     \item $\SimR$ computes and outputs 
%         $\{F_{k_1}(\mathsf{H}(x_i))\}_{x_i \in X}$, 
%         computes $\{F_{k_1}(F_{k_2}(\mathsf{H}(y_i))\}_{y_i \in Y}$, outputs its permutation $\Gamma$.      
% \end{itemize}
% Clearly, $\SimR$'s simulation is identical to the real view. 

% \begin{center}
% \begin{tikzpicture}
% \node[textnode, name = intersection] {$X \cap Y$}; 
% \node[ellipsenode, name = X, left of =intersection, xshift=-1.5em, 
%     minimum height = 3em, minimum width=6em, label={[xshift=-1em]$X$}] {};
% \node[ellipsenode, name = Y, left of =intersection, xshift=1.5em, 
%     minimum height = 3em, minimum width=6em, label={[xshift=1em]$Y$}] {}; 

% \node[textnode, right of = Y, xshift=10em] {$\mathsf{H}(z_i): = h_i \sample D$};
% \end{tikzpicture}
% \end{center}

% \item $\text{Hybrid}_1$: $\SimR$ simulates without the knowledge of $X \cap Y$, 
%     but with the knowledge of $X - X \cap Y$.   
% \begin{itemize}
%     \item $\SimR$ picks $s_j \sample D$ for $j \in [n-m]$ (associated with PRF values for elements in $X - X \cap Y$), 
%         picks $s_k \sample D$ for $k \in [m]$ (implicitly associated with PRF values for elements in $X \cap Y$, 
%         though the exact elememts are unknown), 
%         outputs $(\{s_j\}_{j \in [n-m]}, \{s_k\}_{k \in m})$ according to the order of the final selection vector; 
%         picks $s_\ell \sample D$ for $\ell \in [n-m]$ (associated with PRF values for elements in  $Y - X \cap Y$),  
%         outputs the permutation of $(\{F_{k_2}(s_k)\}_{k \in [m]}, \{F_{k_2}(s_\ell)\}_{\ell \in [n-m]})$.                
% \end{itemize}

% We argue that the view in hybrid 1 and hybrid 2 are computationally indistinguishable. 
% More precisely, a PPT adversary $\mathcal{A}$ \red{(with knowledge of $X$ and $Y$)} against commutative weak PRFs 
% are given $n+m$ tuples $(h_i, s_i)$ where $h_i \sample D$, and is asked to determine if $s_i = F_{k}(h_i)$ or random values. 
% $\mathcal{A}$ implicitly sets $\mathcal{S}$'s randomness $k_1:=k$, picks $k_2 \sample K$. 
% \begin{itemize}
%     \item RO queries: for $z \notin X \cup Y$, picks $h \sample D$ and returns $\mathsf{H}(z):=h$; 
%         for $z_i \in X \cup Y$, returns $\mathsf{H}(z_i) = h_i$. 

%     \item $\mathcal{A}$ prepares $s_j$ for $z_i \in X - X \cap Y$, 
%         $s_k$ for $z_k \in X \cap Y$, 
%         outputs $(\{s_j\}_{j \in [n-m]}, \{s_k\}_{k \in m})$ according to the order of the final selection vector; 
%         prepares $s_\ell \sample D$ for $z_\ell \in Y - X \cap Y$,  
%         outputs the permutation of $(\{F_{k_2}(s_k)\}_{k \in [m]}, \{F_{k_2}(s_\ell)\}_{\ell \in [n-m]})$.               
% \end{itemize}

% If $s_i$ are function values, then the simulation is identical to hybrid 0, 
% else it is identical to hybrid 1. 
% \end{trivlist}
% \end{proof}


% \section{Instantiation Based on the DDH Assumption}\label{sec:protocols}
% We construct a linear complexity PSU protocol from the DDH assumption in Figure~\ref{fig:DH-PSU}.  
% \begin{figure}[!hbtp]
% \begin{framed}
% \begin{minipage}[center]{\textwidth}
% \begin{trivlist}
% \item \textbf{Parameters:} 
% \begin{itemize}
%     \item Common input: $\mathbb{G} = \langle g \rangle$, hash function $\mathsf{H}: \{0,1\}^* \rightarrow \mathbb{G}$. 

%     \item Input of sender $\mathcal{S}$: $X = \{x_1, \dots, x_n\}$.

%     \item Input of receiver $\mathcal{R}$: $Y = \{y_1, \dots, y_n\}$. 
% \end{itemize}

% \item \textbf{Protocol:}

% \begin{enumerate}
% \item $\mathcal{R}$ picks $b \sample \mathbb{Z}_p$, then sends $(\mathsf{H}(y_1)^b, \dots, \mathsf{H}(y_n)^b)$ to $\mathcal{S}$. 

% \item $\mathcal{S}$ picks $a \sample \mathbb{Z}_p$, 
%     sends $(\mathsf{H}(x_1)^a, \dots, \mathsf{H}(x_n)^a)$ to $\mathcal{R}$; 
%     he also computes $(\mathsf{H}(y_1)^b)^a, \dots, \mathsf{H}(y_n)^b)^a)$ and sends its permutation $\Gamma$ to $\mathcal{R}$. 


% \item $\mathcal{R}$ computes $(\mathsf{H}(x_1)^a)^b, \dots, \mathsf{H}(x_n)^a)^b)$, 
%     then sets $v_i = 0$ iff the value is not in $\Gamma$. 

% \item $\mathcal{R}$ with select vector $(v_1, \dots, v_n)$ and $\mathcal{S}$ with input $\{(x_i, \bot)\}_{i \in [n]}$ 
%     engage in one-sided OT. 

% \item $\mathcal{R}$ obtains $X - X \cap Y$, and thus obtains the union $X \cup Y$.  
% \end{enumerate}
% \end{trivlist}
% \end{minipage}
% \end{framed}
% \caption{DH-PSU}\label{fig:DH-PSU}
% \end{figure}

% \begin{theorem}
% The above PSU protocol is secure in the semi-honest model assuming $\mathsf{H}$ is a random oracle and the DDH assumption.
% \end{theorem}

% \begin{proof}
% We exhibit simulators $\SimR$ and $\SimS$ for simulating corrupt $\mathcal{R}$ and $\mathcal{S}$ respectively, 
% and argue the indistinguishability of the produced transcript from the real execution. Let $|X \cap Y| = m$. 


% \begin{trivlist}
% \item \underline{Corrupt sender:} $\SimS$ simulates the view of corrupt $\mathcal{S}$, 
%     which consists of $\mathcal{S}$'s randomness, input, output and received messages.

%     Intuitively, the crux of the security proof is to simulate sender's view without the knowledge of $Y$, 
%     more precisely, the knowledge of $X \cap Y$. 
%     Looking ahead, we will use RO to program the hash outputs for elements in $Y - X \cap Y$ naively, 
%     and program the hash outputs for elements in $X$ in a special way to ensure the associated messages sent from $\mathcal{R}$ 
%     are pseudorandom. 
 
%     We argue the output of $\SimS$ is indistinguishable from the real execution. 
%     For this, we formally show the simulation by proceeding the sequence of hybrid transcripts, 
%     where $T_0$ is the real view of $\mathcal{S}$, and $T_2$ is the output of $\SimS$. 

% \item $\text{Hybrid}_0$: $\SimS$ simulates with the knowledge of $Y$. 
% \begin{itemize}
%     \item $\SimS$ chooses the randomness for $\mathcal{R}$, i.e., picks $b \sample \mathbb{Z}_p$.
    
%     \item RO queries: picks $h_i \sample \mathbb{G}$ and sets $\mathsf{H}(z_i):=h_i$. 

%     \item $\SimS$ outputs $h_i^b$ for $y_i \in Y$.     
% \end{itemize}
% Clearly, $\SimS$'s simulation is identical to the real view. 

% \begin{center}
% \begin{tikzpicture}
% \node[textnode, name = intersection] {$X \cap Y$}; 
% \node[ellipsenode, name = X, left of =intersection, xshift=-1.5em, 
%     minimum height = 3em, minimum width=6em, label={[xshift=-1em]$X$}] {};
% \node[ellipsenode, name = Y, left of =intersection, xshift=1.5em, 
%     minimum height = 3em, minimum width=6em, label={[xshift=1em]$Y$}] {}; 

% \node[textnode, right of = Y, xshift=10em] {$\mathsf{H}(z_i): = h_i \sample \mathbb{G}$};
% \end{tikzpicture}
% \end{center}

% \item $\text{Hybrid}_1$: $\SimS$ still simulates with the knowledge of $Y$, 
%     but slightly change the simulation of random oracle 
% \begin{itemize}
%     \item RO queries: picks $d_i, e_i \sample \mathbb{Z}_p$, sets $\mathsf{H}(z_i):= h_i = h^{d_i} \cdot g^{e_i}$. 
%         Here, $h$ is a fixed random group element.  

%     \item $\SimS$ outputs $h_i^b$ for $y_i \in Y$.            
% \end{itemize}
% The modification in RO simulation does not alter the view. 

% \begin{center}
% \begin{tikzpicture}
% \node[textnode, name = intersection] {$X \cap Y$}; 
% \node[ellipsenode, name = X, left of =intersection, xshift=-1.5em, 
%     minimum height = 3em, minimum width=6em, label={[xshift=-1em]$X$}] {};
% \node[ellipsenode, name = Y, left of =intersection, xshift=1.5em, 
%     minimum height = 3em, minimum width=6em, label={[xshift=1em]$Y$}] {}; 

% \node[textnode, right of = Y, xshift=10em] {$\mathsf{H}(z_i) := h_i = h^{d_i} \cdot g^{d_i}$}; 
% \end{tikzpicture}
% \end{center}

% \item $\text{Hybrid}_2$: $\SimS$ simulates without the knowledge of $Y$, 
%     and changes the simulation in the final input.  
% \begin{itemize}
%     \item $\SimS$ outputs random $f_i$ for $i \in [n]$.       
% \end{itemize}
% We argue that the view in hybrid 2 and hybrid 3 are computationally indistinguishable. 
% More precisely, given $(g, g^\alpha, g^\beta, g^\gamma)$, 
% a PPT adversary $\mathcal{A}$ \red{(with knowledge of $X$ and $Y$)} implicitly sets $\mathcal{R}$'s randomness $b:=\alpha$. 
% \begin{itemize}
%     \item RO queries: sets $h := g^\beta$; picks $d_i, e_i \sample \mathbb{Z}_p$, sets $\mathsf{H}(z_i):=h^{d_i} \cdot g^{e_i}$. 

%     \item $\mathcal{A}$ outputs $g^{\gamma d_i + \alpha e_i}$ for $i \in [n]$.      
% \end{itemize}

% If $(g, g^\alpha, g^\beta, g^\gamma)$ is a DDH tuple, then the simulation is identical to hybrid 1, 
% else it is identical to hybrid 2. 
% \end{trivlist}

% \begin{trivlist}
% \item \underline{Corrupt receiver:} $\SimR$ simulates the view of corrupt $\mathcal{R}$, 
%     which consists of $\mathcal{R}$'s randomness, input, output and received messages.
 
%     We argue the output of $\SimR$ is indistinguishable from the real execution. 
%     For this, we formally show the simulation by proceeding the sequence of hybrid transcripts, 
%     where $T_0$ is the real view of $\mathcal{R}$, and $T_2$ is the output of $\SimR$. 

% \item $\text{Hybrid}_0$: $\SimR$ simulates with the knowledge of $X$. 
% \begin{itemize}
%     \item $\SimR$ chooses the randomness for $\mathcal{S}$, i.e., picks $a \sample \mathbb{Z}_p$.
    
%     \item RO queries: picks $h_i \sample \mathbb{G}$ and sets $\mathsf{H}(z_i):=h_i$. 

%     \item $\SimR$ computes and outputs $\{h_i^{a}\}_{x_i \in X}$; 
%         computes $\{(h_i^{b})^a\}_{y_i \in Y}$, outputs its permutation.      
% \end{itemize}
% Clearly, $\SimR$'s simulation is identical to the real view. 

% \begin{center}
% \begin{tikzpicture}
% \node[textnode, name = intersection] {$X \cap Y$}; 
% \node[ellipsenode, name = X, left of =intersection, xshift=-1.5em, 
%     minimum height = 3em, minimum width=6em, label={[xshift=-1em]$X$}] {};
% \node[ellipsenode, name = Y, left of =intersection, xshift=1.5em, 
%     minimum height = 3em, minimum width=6em, label={[xshift=1em]$Y$}] {}; 

% \node[textnode, right of = Y, xshift=10em] {$\mathsf{H}(z_i): = h_i \sample \mathbb{G}$};
% \end{tikzpicture}
% \end{center}

% \item $\text{Hybrid}_1$: $\SimR$ still simulates with the knowledge of $X$, 
%     but slightly changes the simulation of random oracle 
% \begin{itemize}
%     \item RO queries: for $z_i \notin Y$, picks $r_i \sample \mathbb{Z}_p$ and sets $\mathsf{H}(z_i):= h_i = g^{r_i}$; 
%         for $z_i \in Y$, 
%         picks $d_i, e_i \sample \mathbb{Z}_p$, sets $\mathsf{H}(z_i):= h_i = h^{d_i} \cdot g^{e_i}$. 
%         Here, $h$ is a fixed random group element.  

%     \item $\SimR$ computes and outputs $\{h_i^{a}\}_{x_i \in X}$; 
%         computes $\{(h_i^{b})^a\}_{y_i \in Y}$, outputs its permutation.               
% \end{itemize}
% The modification in RO simulation does not alter the view. 

% \begin{center}
% \begin{tikzpicture}
% \node[textnode, name = intersection] {$X \cap Y$}; 
% \node[ellipsenode, name = X, left of =intersection, xshift=-1.5em, 
%     minimum height = 3em, minimum width=6em, label={[xshift=-1em]$X$}] {};
% \node[ellipsenode, name = Y, left of =intersection, xshift=1.5em, 
%     minimum height = 3em, minimum width=6em, label={[xshift=1em]$Y$}] {}; 

% \node[textnode, right of = Y, xshift=14em] {\(\displaystyle
% \mathsf{H}(z_i) = \left\{ \begin{array}{cl}
% g^{r_i} & \text{if } z_i \notin Y\\
% h^{d_i} \cdot g^{e_i}  & \text{if } z_i \in Y\\ 
% \end{array} \right. \)};  
% \end{tikzpicture}
% \end{center}

% \item $\text{Hybrid}_2$: $\SimR$ simulates without the knowledge of $X$, 
%     and changes the simulation in the final input.  
% \begin{itemize}
%     \item $\SimR$ computes $f_j = h_j^{a}$ for $x_j \in X - X \cap Y$ (note that the information of such $x_j$ 
%         can be inferred from $\mathcal{R}$'s output), picks random $f_k$ for $k \in [m]$ 
%         (implicitly assigned for elements in $X \cap Y$), outputs $(f_1, \dots, f_n)$; 
%         then picks random $f_\ell$ for $\ell \in [n-m]$ (implicitly assigned to elements in $Y - X \cap Y$), 
%         outputs the permutation of $(\{f_k^b\}_{k \in [m]}, \{f_\ell^b\}_{\ell \in [n-m]})$.        
% \end{itemize}
% We argue that the view in hybrid 1 and hybrid 2 are computationally indistinguishable. 
% More precisely, given $(g, g^\alpha, g^\beta, g^\gamma)$, 
% a PPT adversary $\mathcal{A}$ \red{(with knowledge of $X$ and $Y$)} 
% implicitly sets $\mathcal{S}$'s randomness $a:=\alpha$, picks $b \sample \mathbb{Z}_p$. 
% \begin{itemize}
%     \item RO queries: sets $h := g^\beta$; 
%         for $z_i \notin Y$, picks $r_i \sample \mathbb{Z}_p$ and sets $\mathsf{H}(z_i):= h_i = g^{r_i}$; 
%         for $z_i \in Y$, 
%         picks $d_i, e_i \sample \mathbb{Z}_p$, sets $\mathsf{H}(z_i):= h_i = h^{d_i} \cdot g^{e_i}$.  

%     \item $\mathcal{A}$ computes $f_j = (g^\alpha)^{r_i}$ for $x_j \in X - X \cap Y$, 
%         computes $f_k = g^{\gamma d_k + e_k}$ for $k \in [m]$, outputs $(f_1, \dots, f_n)$; 
%         then prepares $f_\ell = g^{\gamma d_\ell + e_\ell}$ for $\ell \in [n-m]$, 
%         outputs the permutation of $(\{f_k^b\}_{k \in [m]}, \{f_\ell^b\}_{\ell \in [n-m]})$.          
% \end{itemize}

% If $(g, g^\alpha, g^\beta, g^\gamma)$ is a DDH tuple, then the simulation is identical to hybrid 1, 
% else it is identical to hybrid 2. 
% \end{trivlist}
% \end{proof}

% \section{Diffie-Hellman Based PSI}
% We recall the ideal PSI functionality in Figure~\ref{fig:fpsi}. 
% \begin{figure}[!hbtp]
% \begin{framed}
% \begin{minipage}[center]{\textwidth}
% \begin{trivlist}
% \item \textbf{Parameters:} 
% \begin{itemize}
% \item Sender $\mathcal{S}$, Receiver $\mathcal{R}$
% \item Set sizes $n_\mathsf{x}$ and $n_\mathsf{y}$
% \end{itemize}

% \item \textbf{Functionality:}
% \begin{itemize}
% \item Wait for input $X=\{x_1,\dots,x_{n_\mathsf{x}}\}\subset \{0,1\}^*$ from the receiver $\mathcal{S}$.
% \item Wait for input $Y=\{y_1,\dots,y_{n_\mathsf{y}}\}\subset \{0,1\}^*$ from the sender $\mathcal{R}$.
% \item Give output $X \cap Y$ to the receiver $\mathcal{R}$.
% \end{itemize}
% \end{trivlist}
% \end{minipage}
% \end{framed}
% \caption{Private Set Intersection Functionality $\FuncPSI$}\label{fig:fpsi}
% \end{figure}

% We first recall the DH-PSI in Figure~\ref{fig:DH-PSI}. 

% \begin{figure}[!hbtp]
% \begin{framed}
% \begin{minipage}[center]{\textwidth}
% \begin{trivlist}
% \item \textbf{Parameters:} 
% \begin{itemize}
%     \item Common input: $\mathbb{G} = \langle g \rangle$ with prime order $p$, 
%         hash function $\mathsf{H}: \{0,1\}^* \rightarrow \mathbb{G}$. 

%     \item Input of sender $\mathcal{S}$: $X = \{x_1, \dots, x_n\}$.

%     \item Input of receiver $\mathcal{R}$: $Y = \{y_1, \dots, y_n\}$. 
% \end{itemize}

% \item \textbf{Protocol:}
% \begin{enumerate}
% \item $\mathcal{S}$ picks $a \sample \mathbb{Z}_p$, 
%     then sends random permutation of $(\mathsf{H}(x_1)^a, \dots, \mathsf{H}(x_n)^a)$ to $\mathcal{R}$. 
% \item $\mathcal{R}$ picks $b \sample \mathbb{Z}_p$, then sends $(\mathsf{H}(y_1)^b, \dots, \mathsf{H}(y_n)^b)$ to $\mathcal{S}$.
% \item $\mathcal{S}$ sends $(\mathsf{H}(y_1)^{ab}, \dots, \mathsf{H}(y_n)^{ab})$ to $\mathcal{R}$. 
% \item $\mathcal{R}$ sets $\Omega = \{\mathsf{H}(x_i)^{ab}\}_{i \in n}$, 
%     and outputs $\{y \mid \mathsf{H}(y)^{ab} \in \Omega\}$.
% \end{enumerate}
% \end{trivlist}
% \end{minipage}
% \end{framed}
% \caption{DH-PSI}\label{fig:DH-PSI}
% \end{figure}

% \begin{theorem}
% The above PSI protocol is secure in the semi-honest model assuming $\mathsf{H}$ is a random oracle and the DDH assumption.
% \end{theorem}

% \begin{proof}
% We exhibit simulators $\SimR$ and $\SimS$ for simulating corrupt $\mathcal{R}$ and $\mathcal{S}$ respectively, 
% and argue the indistinguishability of the produced transcript from the real execution. Let $|X \cap Y| = m$. 

% \begin{trivlist}
% \item \underline{Corrupt sender:} $\SimS$ simulates the view of corrupt $\mathcal{S}$, 
% which consists of $\mathcal{S}$'s randomness, input, output and received messages.
% $\SimS$ proceeds as follows. 

% We now argue the output of $\SimS$ is indistinguishable from the real execution. 
% For this, we formally show the simulation by proceeding the sequence of hybrid transcripts, 
% where $T_0$ is the real view of $\mathcal{S}$, and $T_2$ is the output of $\SimS$. 

% \item $\text{Hybrid}_0$: $\SimS$ simulates with the knowledge of $Y$. 
% \begin{itemize}
%     \item $\SimS$ chooses the randomness for $\mathcal{R}$, i.e., picks $b \sample \mathbb{Z}_p$
    
%     \item RO queries: picks $h_i \sample \mathbb{G}$ and sets $\mathsf{H}(z_i):=h_i$. 

%     \item $\SimS$ outputs $h_i^b$ for $y_i \in Y$ (message in step 2).     
% \end{itemize}
% Clearly, $\SimS$'s simulation is identical to the real view. 

% \begin{center}
% \begin{tikzpicture}
% \node[textnode, name = intersection] {$X \cap Y$}; 
% \node[ellipsenode, name = X, left of =intersection, xshift=-1.5em, 
%     minimum height = 3em, minimum width=6em, label={[xshift=-1em]$X$}] {};
% \node[ellipsenode, name = Y, left of =intersection, xshift=1.5em, 
%     minimum height = 3em, minimum width=6em, label={[xshift=1em]$Y$}] {}; 

% \node[textnode, right of = Y, xshift=10em] {$\mathsf{H}(z_i): = h_i \sample \mathbb{G}$};
% \end{tikzpicture}
% \end{center}

% \item $\text{Hybrid}_1$: $\SimS$ still simulates with the knowledge of $Y$, 
%     but slightly change the simulation of random oracle 
% \begin{itemize}
%     \item RO queries: picks $d_i, e_i \sample \mathbb{Z}_p$, 
%         sets $\mathsf{H}(z_i):= h_i = h^{d_i} \cdot g^{e_i}$. 
%         Here, $h$ is a random group element fixed at beginning.  

%     \item $\SimS$ outputs $h_i^b$ for $y_i \in Y$.            
% \end{itemize}
% The modification in RO simulation does not alter the view. 

% \begin{center}
% \begin{tikzpicture}
% \node[textnode, name = intersection] {$X \cap Y$}; 
% \node[ellipsenode, name = X, left of =intersection, xshift=-1.5em, 
%     minimum height = 3em, minimum width=6em, label={[xshift=-1em]$X$}] {};
% \node[ellipsenode, name = Y, left of =intersection, xshift=1.5em, 
%     minimum height = 3em, minimum width=6em, label={[xshift=1em]$Y$}] {}; 

% \node[textnode, right of = Y, xshift=10em] {$\mathsf{H}(z_i) := h_i = h^{d_i} \cdot g^{e_i}$}; 
% \end{tikzpicture}
% \end{center}

% \item $\text{Hybrid}_2$: $\SimS$ simulates without the knowledge of $Y$, 
%     and changes the simulation in the final input.  
% \begin{itemize}
%     \item $\SimS$ outputs random $f_i$ for $i \in [n]$.       
% \end{itemize}
% We argue that the view in hybrid 2 and hybrid 3 are computationally indistinguishable based on the DDH assumption. 
% More precisely, given $(g, g^\alpha, g^\beta, g^\gamma)$, 
% a PPT adversary $\mathcal{A}$ \red{(with knowledge of $X$ and $Y$)} implicitly sets $\mathcal{R}$'s random tape $b:=\alpha$: 
% \begin{itemize}
%     \item RO queries: sets $h = g^\beta$; picks $d_i, e_i \sample \mathbb{Z}_p$, 
%         sets $\mathsf{H}(z_i):=h_i = h^{d_i} \cdot g^{e_i}$. 

%     \item $\mathcal{A}$ outputs $g^{\gamma d_i + \alpha e_i}$ for $i \in [n]$.      
% \end{itemize}

% If $(g, g^\alpha, g^\beta, g^\gamma)$ is a DDH tuple, 
% then the simulation is identical to hybrid 1 because $g^{\gamma d_i + \alpha e_i} = g^{\alpha \beta d_i + \alpha e_i} = h_i^b$, 
% else it is identical to hybrid 2. 
% \end{trivlist}

% \begin{remark}
% From hybrid 1, we program $\mathsf{H}(z_i) = h^{d_i} \cdot g^{e_i}$ 
% to obtain a tight reduction by leveraging the random self-reducible property of the DDH assumption. 
% Alternatively, we can simply program $\mathsf{H}(z_i) = h^{d_i}$, then argue the indistinguishability using hybrid argument. 
% As a result, the security reduction comes with a loose factor $n$.    
% \end{remark}

% \begin{trivlist}
% \item \underline{Corrupt receiver:} $\SimR$ simulates the view of corrupt $\mathcal{R}$, 
% which consists of $\mathcal{R}$'s randomness, input, output and received messages. $\SimR$ proceeds as follows. 

% We now argue the output of $\SimR$ is indistinguishable from the real execution. 
% For this, we formally show the simulation by proceeding the sequence of hybrid transcripts, 
% where $T_0$ is the real view of $\mathcal{R}$, and $T_2$ is the output of $\SimR$. 

% The simulator has to emulate two messages: 
% \begin{itemize}
%     \item message in step 1: a permutation of $(\mathsf{H}(x_1)^a, \dots, \mathsf{H}(x_n)^a)$
%     \item message in step 3: $(\mathsf{H}(y_1)^{ab}, \dots, \mathsf{H}(y_n)^{ab})$
% \end{itemize}
% Since the elements in $X - X \cap Y$ are unknown, 


% \item $\text{Hybrid}_0$: $\SimR$ simulates with the knowledge of $X$. 
% \begin{itemize}
%     \item $\SimR$ chooses the randomness for $\mathcal{S}$, i.e., picks $a \sample \mathbb{Z}_p$
    
%     \item RO queries: picks $h_i \sample \mathbb{G}$ and sets $\mathsf{H}(z_i):=h_i$. 

%     \item $\SimR$ outputs a random permutation of $(\mathsf{H}(x_1)^a, \dots, \mathsf{H}(x_n)^a)$ and 
%         $(\mathsf{H}(y_1)^{ab}, \dots, \mathsf{H}(y_n)^{ab})$.     
% \end{itemize}
% Clearly, $\SimR$'s simulation is identical to the real view. 

% \begin{center}
% \begin{tikzpicture}
% \node[textnode, name = intersection] {$X \cap Y$}; 
% \node[ellipsenode, name = X, left of =intersection, xshift=-1.5em, 
%     minimum height = 3em, minimum width=6em, label={[xshift=-1em]$X$}] {};
% \node[ellipsenode, name = Y, left of =intersection, xshift=1.5em, 
%     minimum height = 3em, minimum width=6em, label={[xshift=1em]$Y$}] {}; 

% \node[textnode, right of = Y, xshift=10em] {$\mathsf{H}(z_i): = h_i \sample \mathbb{G}$};
% \end{tikzpicture}
% \end{center}

% \item $\text{Hybrid}_1$: $\SimR$ still simulates with the knowledge of $X$, 
%     but slightly change the simulation of random oracle 
% \begin{itemize}
%     \item RO queries: for $z_i \in Y$, picks $r_i \sample \mathbb{Z}_p$ 
%         and sets $\mathsf{H}(z_i):= h_i = g^{r_i}$; for $z_i \notin Y$, 
%         picks $d_i, e_i \sample \mathbb{Z}_p$, sets $\mathsf{H}(z_i):= h_i = h^{d_i} \cdot g^{e_i}$. 
%         Here, $h$ is a random group element fixed at beginning.              
% \end{itemize}
% The modification in RO simulation does not alter the view. 

% \begin{center}
% \begin{tikzpicture}
% \node[textnode, name = intersection] {$X \cap Y$}; 
% \node[ellipsenode, name = X, left of =intersection, xshift=-1.5em, 
%     minimum height = 3em, minimum width=6em, label={[xshift=-1em]$X$}] {};
% \node[ellipsenode, name = Y, left of =intersection, xshift=1.5em, 
%     minimum height = 3em, minimum width=6em, label={[xshift=1em]$Y$}] {}; 

% \node[textnode, right of = Y, xshift=14em] {\(\displaystyle
% \mathsf{H}(z_i) = \left\{ \begin{array}{cl}
% g^{r_i} & \text{if } z_i \in Y\\
% h^{d_i} \cdot g^{e_i}  & \text{if } z_i \notin Y\\ 
% \end{array} \right. \)}; 
% \end{tikzpicture}
% \end{center}

% \item $\text{Hybrid}_2$: $\SimR$ simulates without the knowledge of $X$, 
%     and changes the simulation in the output.  
% \begin{itemize}
%     \item $\SimR$ picks random $f_j$ for $j \in [n-m]$ (associated with elements in $X - X \cap Y$), 
%         prepares $f_k = h_k^a$ for $k \in [m]$ (associated with elements in $X \cap Y$), 
%         then outputs a random permutation of $\{f_i\}_{i \in [n]}$, 
%         and $(\mathsf{H}(y_1)^{ab}, \dots, \mathsf{H}(y_n)^{ab})$.       
% \end{itemize}
% We argue that the view in hybrid 2 and hybrid 3 are computationally indistinguishable based on the DDH assumption. 
% More precisely, given $(g, g^\alpha, g^\beta, g^\gamma)$, 
% a PPT adversary $\mathcal{A}$ \red{(with knowledge of $X$ and $Y$)} 
% implicitly sets $\mathcal{S}$'s randomness $a:=\alpha$, picks $\mathcal{R}$'s randomness $b \sample \mathbb{Z}_p$: 
% \begin{itemize}
%     \item RO queries: sets $h = g^\beta$; for $z_i \in Y$, picks $r_i \sample \mathbb{Z}_p$ 
%         and sets $\mathsf{H}(z_i):= h_i = g^{r_i}$; for $z_i \notin Y$, 
%         picks $d_i, e_i \sample \mathbb{Z}_p$, sets $\mathsf{H}(z_i):= h_i = h^{d_i} \cdot g^{e_i}$.  

%     \item $\mathcal{A}$ computes $f_j = g^{\gamma d_j + \alpha e_j}$ for $j \in [n-m]$, 
%         $f_k = (g^\alpha)^{r_k}$ for $k \in [m]$, 
%         then outputs a random permutation of $\{f_i\}_{i \in n}$ 
%         and $\{\mathsf{H}(y_1)^{ab}, \dots, \mathsf{H}(y_n)^{ab}\}$, where $\mathsf{H}(y_i)^{ab} = (g^\alpha)^{r_ib}$.      
% \end{itemize}

% If $(g, g^\alpha, g^\beta, g^\gamma)$ is a DDH tuple, 
% then the simulation is identical to hybrid 1 because $g^{\gamma d_j + \alpha e_j} = g^{\alpha \beta d_i + \alpha e_i} = h_j^a$, 
% else it is identical to hybrid 2. 
% \end{trivlist}
% \end{proof}
